\documentclass{article}

\usepackage[colorlinks]{hyperref}
% \usepackage{enumitem}
% \setlist[description]{style=nextline}

\title{Simulations for NaLoDuCo experimentation with reinforcement learning
agents}

\author{Joaqu\'{i}n Rapela}

\begin{document}

\maketitle

\section{Motivations}

In physics, engineering and robotics simulations are widely used to test, debug
and optimize experiments before real wold trials. However, neuroscience
experiments lack equivalent tools.

In short-duration experiments one can iterate over several experimental
configurations quickly to find the best one.
%
This possibility in NaLoDuCo experiments is drastically limited.
%
Therefore, it is critical to have an environment to simulate in detail NaLoDuCo
experiments with realistically behaving agents to allow scientists to quickly
try different experimental setups.
%
Simulations can save high costs of iterations. Every failed experiment means
months of lost work, as well as the cost of caring for animals and maintaining
experimental setups.

\section{Proposed solution}

We propose to develop a realistic, detailed simulation framework that allows
scientists to pretest their foraging experiments before running them on real
animals.

\section{Key features of the simulation}

\begin{description}
    \item[Reinforcement Learning-Based Virtual Mice]\leavevmode\\

        \begin{enumerate}

            \item We will use RL agents to simulate mouse behavior in foraging tasks.

            \item Agents will be trained using prior data and reward structures that match real foraging constraints.

            \item Previous work has developed RL-based foraging models (e.g.,
                \href{https://arxiv.org/abs/2210.08085}{arXiv:2210.08085}), which we can build upon.  

        \end{enumerate}

    \item[Realistic Environmental and Experimental Constraints]\leavevmode\\

        \begin{enumerate}

            \item The simulation will include environmental factors (e.g., food
                availability, maze layouts, reward schedules).

            \item We will incorporate biological constraints (e.g., circadian
                rhythms, fatigue, adaptation).

            \item The system will model sensor noise and drift to match
                real-world data collection.

        \end{enumerate}

    \item[Interactive Experiment Design and Debugging]\leavevmode\\

        \begin{enumerate}

            \item Users can customize parameters of the experiment (e.g.,
                reward schedules, number of patches, number of nets, number,
                resolution and positions of cameras).

            \item RL agents will be visualised as realistic mice and their
                motion will be recorded by the different cameras, at the
                corresponding resolutions and frame rates. These recordings
                will allow to run body part tracking and pose estimation
                algorithms on simulated data, and test if the cameras
                specification and layout allows required estimation
                accuracy.

            \item Scientists will be able to run and iterate simulations
                quickly, testing different hypotheses before committing to real
                experiments.

            \item The platform will provide detailed logs, making it easier to
                diagnose potential pitfalls in experiment design.

        \end{enumerate}

    \item[Data-Driven Model Updating]\leavevmode\\
        \begin{enumerate}

            \item As new foraging experimental data is collected, it will be
                used to update the simulation for better accuracy.

            \item This allows for a continual refinement of the RL-based
                foraging models.

        \end{enumerate}

\end{description}

\section{How can simulations help NaLoDuCo experimentation?}

If I want to test if the validity of Marginal Value Theorem in a NaLoDuCo
experiment setup is it better to use one or two patches?, what food reward rate
should we use?, what inter-patch distance should I set?, would it be helpful to
introduce predators? I could quickly address all these questions in realistically
simulated the foraging experiments.

Does foraging become more efficient over time? If so, how long should an
experiment be to reveal this increase in efficiency? We could address these
questions with an optimal RL foraging agent.

Is an animal behavior adapting over time? To address this question I can
estimate parameters of optimal RL models at different points in time and check
if these parameters change significantly.

\end{document}
