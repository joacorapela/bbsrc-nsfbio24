
\subsection{Relevance}

Most experiments in Neuroscience are controled by predefined rules or by simple
neural and/or behavioral observations. Here we propose to control neuroscience
experiments using advanced inferences on behavioral and/or neural data.

Intelligent experimental control is relevant to long-duration naturalistic
experimentation for at least two reasons. First the extremely large size of
recorded dataset may forbid storing all raw data and online machine learning
algorithms can help decide what data to store. For instance, we want to record
the behaviour of multiple animals in large arenas with high resolution. This
requires using multiple high-definition cameras to record videos of different
parts of the arena. It is not feasible to store the videos by all cameras in
long duration experiments. To overcome this problem, we can use probabilistic
machine learning methods to track online the positions of the mice in the
arena. When the tracking confidence of this methods is high, we would only save
the highresolution videos of the cameras filming mice, but when their
confidence is low we would save the videos of all cameras.

Second, we want to make neural interventions informed by online inferences from
machine learning methods. For example, in a foraging experiment we could find
that a neural latent variable peaks before the instant when a mouse starts
accelerating to leave the current patch (the latent variable could be estimated
using a Poisson linear dynamical system and the acceleration using a linear
dynamical system). We could hypothesise that the neural population associated
with the latent variable is responsible for the decision of leaving a patch.
We could then test this hypothesis by optogenetically silencing this population
while an animal is on a patch and checking if it leaves the patch or not.

\subsection{Deliverables}

\begin{enumerate}

    \item software, hardware and papers demonstrating applications of
        intelligent experimental control in neuroscience experimentation.

\end{enumerate}

\subsection{Previous work}

\begin{itemize}

    \item Bonsai is an excellent software for close-loop
        neuroscience experimental control.

    \item funded by a BBSRC
        award\footnote{\url{https://gow.bbsrc.ukri.org/grants/AwardDetails.aspx?FundingReference=BB\%2FW019132\%2F1}}
        we are integrating machine learning methods into Bonsai.

\end{itemize}

\subsection{Future work}

\begin{itemize}

    \item Bonsai has been used for close-loop control using direct
        observations. With the advanced machine learning methods that we have
        added to Bonsai, we can know infer subtle patterns in behavioral and
        neural recordings not visible to the nacked eye. Now we will find
        applications where these patterns can be used for experimental control.
        Fortunately, we work at the SWC, where such applications abound.

    \item it would be very useful to perform control based on patterns inferred
        from spiking activity recorded by Neuropixels probes. To do so, we need
        online spike sorting methods for Neuropixels probes. These methods are
        underdeveloped and, if neccessary, we will need to develop one.

\end{itemize}
