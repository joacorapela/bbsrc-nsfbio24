\documentclass{beamer}

\usetheme{Berkeley} % Sidebar theme
% \setbeamertemplate{headline}{} % Removes title from sidebar
% \setbeamertemplate{sidebar left}{\insertsectionnavigation{.5\textwidth}{\hskip-1em}{}} % Removes author info

\usepackage{xcolor}
% Define custom colors (optional)
\definecolor{myblue}{RGB}{0, 0, 200} % Dark blue
\definecolor{myred}{RGB}{200, 0, 0}  % Dark red

% Set link colors
\hypersetup{
    colorlinks=true,
%     linkcolor=myblue,    % Color for internal links (e.g., table of contents)
    urlcolor=myred,      % Color for URLs
    citecolor=green      % Color for citations (if using BibTeX)
}

% \usepackage{apalike}
\usepackage{natbib}

%Information to be included in the title page:
\title{Deliverables, Previous and Future Work}
\author{Joaquin Rapela}
\institute{Gatsby Computational Neuroscience Unit}
\date{\today}

\AtBeginSection[]
{
 \begin{frame}<beamer>
 \frametitle{Contents}
 \tableofcontents[currentsection]
 \end{frame}
}

\begin{document}

\frame{\titlepage}

\begin{frame}
    \frametitle{Contents}

    \tableofcontents

\end{frame}

\section{Aims}

\begin{frame}
    \frametitle{Aims}

\begin{enumerate}

    \item cross-fertilisation between the SWC/GCNU and the AIND in NaLoDuCo
        experimentation.

    \item dissemination to the research community of hardware and software
        technology for NaLoDuCo experimentation.

\end{enumerate}

\end{frame}

\section{Data acquisition, management, quality control and alerts}

\begin{frame}
    \frametitle{Deliverables}

    \begin{enumerate}

        \item hardware specifications for recordings of behaviour and neural
        activity used at the
        SWC/GCNU\footnote{\url{https://sainsburywellcomecentre.github.io/aeon\_docs/reference/hardware.html}}
        and at the AIND for head-fixed
        foraging\footnote{\url{https://www.allenneuraldynamics.org/platforms/behavior}}
        and freely moving odour exploration.

        \item software for managing long-duration recordings (e.g., data storage, data
        indexing).

        \item software for online/offline quality control.

        \item software for creating alerts.

        \item software for online (behavioral and neural) data visualisation.

        \item software for online (behavioral and neural) data analysis.

    \end{enumerate}

\end{frame}

\begin{frame}
    \frametitle{Previous work}

    \begin{itemize}
   
        \only<1>{
        \item the SWC has performed foraging experiments
    
        \begin{itemize}
    
            \item lasting xx weeks and recording behaviour only
            \item lasting yy weeks and recording behaviour and electrophysiology
            \item data is stored in files and in a MySQL database
    
        \end{itemize}
    
        \item the AIND has performed foraging experiments in head-fixed mice.
            These experiments are a few hours long.
    
        \item the AIND is setting up the odour exploration experiments that
            will last several days.
    
        \item  items 1--2: above have been completed for the SWC foraging experiments
    
        \item item 3: the SWC has developed several quality control for behavior.
    
        \item item 4: alerts have been developed for behaviour and ephys.
        }
        \only<2>{
        \item  item 5: the SWC has developed some online behavioural data
            visualisation tools in Bonsai.
        \item  item 6: funded by BBSRC, we have integrated into Bonsai tools for online data
      analysis:
    
        \begin{itemize}
    
            \item estimate kinematics of mice
    
            \item estimate kinematic states of mice using Hidden Markov Models
    
            \item clusterless point-process decoder of mice position and replay
                from ephys recordings.
    
        \end{itemize}
    
        \item  disseminated documentation of hardware used at the AIND to perform
            head-fixed foraging
            exeperiments\footnote{\url{https://www.allenneuraldynamics.org/platforms/behavior}}
            in virtual reality setups.
    
        \item  disseminated documentation on software used at the SWC to control
            NaLoDuCo foraging experiments (see
            \href{https://github.com/SainsburyWellcomeCentre/aeon_mecha.}{repo})
        }
        \only<3>{
    
        \item  disseminated documentation on machine learning methods integrated
            into Bonsai for analyzing behavioral and neural time series in real time
            (see \href{https://bonsai-rx.org/machinelearning/.}{repo})
        }    
    \end{itemize}

\end{frame}

\begin{frame}
    \frametitle{Future work}

    \begin{itemize}
   
        \item item 3: the SWC and the AIND have developed several
        tools for offline quality control. Next, we need to build online
        versions of them.

        \item item 4: develop more software for data visualisation.

        \item items 5 and 6: develop more software for online data analysis.

        \begin{itemize}

            \item online estimate of latent variables from Neuropixels
            recordings.

            \item online estimate of RL models.

        \end{itemize}

        \item assist Dr.~Carl Schoonover (AIND) on the use hardware and
        software developed at the SWC/GCNU to create olfactory learning
        NaLoDuCo experiments.

    \end{itemize}

\end{frame}

\section{Data analysis}

\begin{frame}
    \frametitle{Deliverables}

    \begin{enumerate}

        \item methods to analyse behavioural and electrophysiological
        recordings from NaLoDuCo experiments that are \textbf{online} and
        \textbf{adaptive to non-stationarity} in measurements.

        \begin{scriptsize}
        For behavioural data, we will investigate methods to
    
        \begin{itemize}
    
			\begin{scriptsize}
            \item track multiple body parts of animals (deep neural networks)
    
            \item infer kinematics of foraging mice (linear dynamical systems)
    
            \item segment behaviour into discrete states
    
            \item characterize short- and long-term periodicities in behavior
    
            \item infer the rules that govern mice behaviour from behavioural
            observations only] (i.e., policy inference).
			\end{scriptsize}
    
        \end{itemize}
    
        For neural data, we will investigate methods to:
        \begin{itemize}
    
			\begin{scriptsize}
            \item estimate low-dimensional continual representations of neural
            activity (i.e., latents inference)
    
            \item segment neural activity into discrete states
    
            \item characterize short- and long-term periodicities in neural
                activity
    
            \item decode environment variables from neural activity
			\end{scriptsize}
    
        \end{itemize}
        \end{scriptsize}
    
        \item integration of these methods into DANDI, as in Dendro, so that users
        can run them on NaLoDuCo datasets stored in DANDI.

    \end{enumerate}

\end{frame}

\begin{frame}
    \frametitle{Previous work}

    \begin{enumerate}

		\item At the Gatsby we have invented several methods for the
		characterisation of neural time
		series~\citep[e.g.,][]{yuEtAl09,dunckerAndSahani18,ruttenEtAl20,yuEtAl24,buesingEtAl12a,buesingEtAl12b,mackeEtAl15,soulatEtAl21,walkerEtAl23,turnerAndSahani14,osheaEtAl22}

		\item \href{https://github.com/magland/dendro}{Dendro} allows to
		perform advanced data analysis on DANDI. It allows to reuse previous
		analysis.

    \end{enumerate}

\end{frame}

\bibliographystyle{apalike}
\bibliography{latentsVariablesModels,neuralManinuplation,rpm,stateSpaceModels,spectralAnalysis,tensorDecomposition}

\end{document}
