
\documentclass[12pt]{article}

\usepackage{graphicx}
\usepackage{hyperref}
\usepackage{bibunits}
\usepackage{longtable}
\usepackage{natbib}
\usepackage{titlesec}
\hypersetup{
    colorlinks = true,
    linkcolor = {green},
}
\usepackage{verbatim}
\usepackage{lineno}

\newenvironment{instruction}{\par\color{red}}{\par}

% begin add subsubsubsection
\titleclass{\subsubsubsection}{straight}[\subsection]

\newcounter{subsubsubsection}[subsubsection]
\renewcommand\thesubsubsubsection{\thesubsubsection.\arabic{subsubsubsection}}
\renewcommand\theparagraph{\thesubsubsubsection.\arabic{paragraph}} % optional; useful if paragraphs are to be numbered

\titleformat{\subsubsubsection}
  {\normalfont\normalsize\bfseries}{\thesubsubsubsection}{1em}{}
\titlespacing*{\subsubsubsection}
{0pt}{3.25ex plus 1ex minus .2ex}{1.5ex plus .2ex}

\makeatletter
\renewcommand\paragraph{\@startsection{paragraph}{5}{\z@}%
  {3.25ex \@plus1ex \@minus.2ex}%
  {-1em}%
  {\normalfont\normalsize\bfseries}}
\renewcommand\subparagraph{\@startsection{subparagraph}{6}{\parindent}%
  {3.25ex \@plus1ex \@minus .2ex}%
  {-1em}%
  {\normalfont\normalsize\bfseries}}
\def\toclevel@subsubsubsection{4}
\def\toclevel@paragraph{5}
%\def\toclevel@paragraph{6}
\def\toclevel@subparagraph{6}
\def\l@subsubsubsection{\@dottedtocline{4}{7em}{4em}}
\def\l@paragraph{\@dottedtocline{5}{10em}{5em}}
\def\l@subparagraph{\@dottedtocline{6}{14em}{6em}}
\makeatother

\setcounter{secnumdepth}{4}
\setcounter{tocdepth}{4}
% end add subsubsubsection

\linenumbers

\title{Enabling Naturalistic, Long-Duration and Continual Experimentation with
Advanced Machine Learning Methods}

\begin{document}
\defaultbibliography{longDurationExperimentation,neuroEthology,machineLearning,signalProcessing,bonsai}
\defaultbibliographystyle{apalike}

\tableofcontents

\pagebreak

\section{Summary}


\begin{instruction}
Word limit: 550

In plain English, provide a summary we can use to identify the most suitable
experts to assess your application.

We usually make this summary publicly available on external-facing websites,
therefore do not include any confidential or sensitive information. Make it
suitable for a variety of readers, for example:

\begin{itemize}
    \item opinion-formers
    \item policymakers
    \item the public
    \item the wider research community
\end{itemize}

\paragraph{Guidance for writing a summary}

Clearly describe your proposed work in terms of:

\begin{itemize}
    \item context
    \item the challenge the project addresses
    \item aims and objectives
    \item potential applications and benefits
    \item its relevance to the
    \href{https://www.ukri.org/publications/bbsrc-strategic-delivery-plan/}{BBSRC long-term research and innovation priorities}
    and, if applicable
    \href{https://www.ukri.org/councils/bbsrc/remit-programmes-and-priorities/our-research-portfolio-and-priorities/responsive-mode-spotlights/}{Responsive Mode Spotlight areas}

\end{itemize}
\end{instruction}



\pagebreak
\section{Core team}


\begin{instruction}
List the key members of your team and assign them roles from the following:

\begin{itemize}
    \item project lead (PL)
    \item project co-lead (UK) (PcL)
    \item specialist
    \item professional enabling staff
    \item research and innovation associate
    \item technician
    \item researcher co-lead (RcL)
\end{itemize}

Only list one individual as project lead.

UKRI has introduced a new addition to the ‘specialist’ role type. Public
contributors such as people with lived experience can now be added to an
application.

Find out more about
\href{https://www.ukri.org/publications/roles-in-funding-applications/roles-in-funding-applications-eligibility-responsibilities-and-costings-guidance/}{UKRI’s
core team roles in funding applications} and our
\href{https://www.ukri.org/councils/bbsrc/guidance-for-applicants/check-if-youre-eligible-for-funding/applicants-and-co-applicants/}{eligibility
guidance}.

\end{instruction}


\pagebreak
\section{Application questions}

\subsection{BBSRC schemes}


\begin{instruction}
Word limit: 1

Indicate the scheme through which you are applying.

In the text box, copy the number corresponding to the scheme you are applying
through. These are:

\begin{enumerate}
    \item standard (no scheme)
    \item Industrial Partnership Award (IPA)
    \item LINK
    \item Brazil (FAPESP)
    \item Luxembourg (FNR)
    \item NSF-Bio
\end{enumerate}

Additional guidance

This is for administrative purposes to help the initial application processing.

Please follow the scheme specific guidance below and upload the additional
documents listed as a single PDF no larger than 8MB:

IPA or LINK:

\begin{itemize}

    \item a letter from your institution’s technology transfer office outlining the
management of outputs from the proposed research

\end{itemize}

FAPESP:

\begin{itemize}

    \item FAPESP proposal form
    \item FAPESP consolidated budget form
    \item FAPESP letter of eligibility

\end{itemize}

FNR:

\begin{itemize}

    \item CVs of international collaborators
    \item FNR ‘INTER’ budget form
    \item FNR ‘INTER’ cost justification

\end{itemize}

NSF-Bio:

\begin{itemize}

    \item US biosketches
    \item US budget forms

\end{itemize}
\end{instruction}


\pagebreak
\subsection{BBSRC remit classification}


\begin{instruction}
Word limit: 1

Your application will be considered by one of our four research committees
made up of independent experts. Indicate which you feel would be best placed
to assess your application.

In the text box, write only the letter (in uppercase) corresponding to the
committee you feel would be best placed to assess your application. These are:

\begin{description}

    \item[A] animal disease, health and welfare

    \item[B] plants, microbes, food and sustainability

    \item[C] genes, development, and science, technology, engineering and maths
    (STEM) approaches to biology

    \item[D] molecules, cells and industrial biotechnology

\end{description}

Additional guidance:

This is for administrative purposes to help the initial application processing.
We will check your choice and make a final decision as to which committee will
assess your application.

\end{instruction}


\pagebreak
\subsection{Vision}
\begin{instruction}
Word limit: 550

What are you hoping to achieve with your proposed work?

What the assessors are looking for in your response

Explain how your proposed work:

\begin{enumerate}

    \item is of excellent quality and importance within or beyond the field(s) or area(s)

    \item has the potential to advance current understanding, or generate new
knowledge, thinking or discovery within or beyond the field or area

    \item is timely given current trends, context, and needs

    \item impacts world-leading research, society, the economy, or the environment

\end{enumerate}

You may demonstrate elements of your responses in visual form if relevant.
Further details are provided in the Funding Service.
References may be included within this section.
\end{instruction}

\begin{bibunit}
\subsubsection{Context}

Conventional systems neuroscience experiments are typically short in duration
and often place significant constraints on subjects behaviours to simplify data
analysis.
%
However, these restrictions may limit our ability to observe critical
aspects of brain function and behaviour that only manifest in more naturalistic
and extended conditions.

At the Sainsbury Wellcome Centre (SWC) and Gatsby Computational Neuroscience
Unit (GCNU) we are pioneering Naturalistic, Long-Duration, and Continual
(NaLoDuCo) foraging experiments in mice that span weeks to months. During these
experiments, we collect high-resolution behavioural and neural recordings in
naturalistic settings.

This novel  approach will enable researchers to explore neural mechanisms
underlying ethological behaviours in naturalistic environments over months, for
the first time.  The experiments will shed new light on a wide range of poorly
understood neural mechanisms, including how the brain structures complex
behavioural sequences as a function of the animal needs, learning and social
dynamics.
%
The data generated from NaLoDuCo experiments represent an entirely new resource
in neuroscience, with the potential to drive breakthroughs and discoveries that
are beyond the reach of traditional experiments.

While experiments in neuroscience that are naturalistic, long-duration, or
continuous have been conducted in the past (e.g.,
[\href{https://pubmed.ncbi.nlm.nih.gov/37656619/}{1}]), to the best of our
knowledge, we are the first to integrate all three of these features in a
single experimental paradigm.
%
Experiments of this type have been advocated by experts in the field years ago
([\href{https://pubmed.ncbi.nlm.nih.gov/31600508/}{2}], p.19), yet they have
not been implemented so far.

The central goal of the Allen Institute for Neural Dynamics (AIND), our US
collaborator, is to understand how the brain works at the level of individual
neurons.
%
Central to its mission is the development of neuro-technologies to acquire and
distribute massive ammounts of neural data.
%
They are also investigating foraging behavior, but using head-fixed mice.

% Since the project started in 2021, our UK business partner, NeuroGEARS Ltd.\
% has been contracted by the SWC to lead the implementation of the NaLoDuCo
% experimental framework. It also provides services to the AIND.

The extremely large datasets--on the order of hundreds of terabytes--gathered
from experiments spanning weeks to months pose significant challenges in data
acquisition, management, distribution, visualisation, and analysis.
%
Together, the GCNU, SWC and AIND will address these challenges, co-develop this
new type of experimentation, share expertise and build software infrastructure
to help scientists around the world perform NaLoDuCo experiments.

\subsubsection{Focus areas}

Developing platform technologies for:

\begin{description}

    \item[Experimental Control, Data Acquisition \& Management] Controlling
        sophisticated experiments and efficiently gathering and organising
        massive datasets over extended periods.

    \item[Data Sharing] Providing global access to large-scale datasets.

    \item[Data Visualisation] Building web-based visualisation of very large
        behavioural and neural data.

    \item[Data Analysis] Characterising behavioural and neural recordings with
        advanced machine learning methods.

\end{description}

\subsubsection{Specific benefits from our collaboration}

The foraging experiments at the AIND are different from those at the SWC. They
do not probe freely moving and naturalistic behaviour, but are able to perform
electrophysiological recordings more densely than those at the SWC.
%
These experimental approaches to foraging are complementary and this
collaboration will greatly benefit both of them.

Currently, both GCNU and AIND are independently developing methods to address
the previous focus areas. We will join forces to co-develop these methods and
our foraging research programs, leveraging our combined expertise for greater
impact.

\putbib
\end{bibunit}

\pagebreak
\subsection{Approach}

\begin{instruction}

Word limit: 3,300

How are you going to deliver your proposed work?

What the assessors are looking for in your response

Explain how you have designed your approach so that it:

\begin{enumerate}

    \item is effective and appropriate to achieve your objectives

    \item is feasible, and comprehensively identifies any risks to delivery and
    how they will be managed

    \item uses a clearly written and transparent methodology (if applicable)

    \item summarises the previous work and describes how this will be built
    upon and progressed (if applicable)

    \item will maximise translation of outputs into outcomes and impacts

    \item describes how your, and if applicable your team’s, research
    environment (in terms of the place and relevance to the project) will
    contribute to the success of the work

\end{enumerate}

You may demonstrate elements of your responses in visual form if relevant.

Please make sure to check sizing and readability of the image using ‘read view’
prior to submission. Further details are provided in the Funding Service.

References may be included within this section.

Within the ‘Approach’ section we also expect you to:

\begin{itemize}

    \item provide a detailed and comprehensive project plan including
    milestones and timelines in the form of an embedded Gantt chart or similar
    (please make sure to check sizing and readability of the image using ‘read
    view’ prior to submission)

\end{itemize}

BBSRC’s
\href{https://www.ukri.org/publications/bbsrc-equality-diversity-and-inclusion-action-plan/bbsrc-action-plan-for-equality-diversity-and-inclusion-in-the-biosciences-2022-to-2025/}{action
plan for EDI} outlines our commitment to removing barriers to participation in
our programmes, ensuring investments do not inadvertently prevent access or
usage by individuals from minority groups, for example disabled researchers.

To this end, applications should identify how accessibility and inclusiveness
in the widest sense have been incorporated into the design of the project. For
example, you may wish to reference relevant institutional strategies and
policies which support equality, diversity, and inclusion as they relate to
access to equipment and facilities and indicate how the proposed project has
been designed and will be delivered with broad access in mind.

\end{instruction}

\begin{bibunit}
\subsubsection{Data collection \& management}

We have developed an innovative platform for housing of mice in large arenas
(\textgreater 2m diameter) enabling precise behavioural manipulation and
high-resolution monitoring
(\href{https://www.gatsby.ucl.ac.uk/~rapela/bbsrc\_nsfbio/figures/foragingArena.png}{online
figure},
[\href{https://www.abstractsonline.com/pp8/?_gl=1*it0gi6*_gcl_au*MTUyNDE0NDQwLjE3Mjc2OTgyODM.*_ga*MTUxNDI2NDg5LjE3Mjc2OTgyODM.*_ga_T09K3Q2WDN*MTcyOTUwNDUzNy4yLjEuMTcyOTUwNDY3Ny41NC4wLjA.#!/20433/presentation/22271}{2}]).
%
We have openly shared software for supporting data
acquisition~[\href{https://github.com/SainsburyWellcomeCentre/aeon_acquisition}{3}]
and
management~[\href{https://github.com/SainsburyWellcomeCentre/aeon_mecha}{4}] in this
arena.
%
Additionally, the platform supports continuous, long term monitoring of neural
activity with Neuropixels probes, capable of recording from thousands of
neurons simultaneously spanning the entire brain depth.
%
This setup has allowed us to collect several week-long datasets with single and
multiple mice per arena.

To facilitate the replication of our experimental setup by other groups, we
will share instructions for building foraging arenas, as well as specifications
of hardware used in them,
%
and we will improve the documentation of the software repositories for data
acquisition and management.

\subsubsection{Sharing data and methods}

The very large datasets produced by NaLoDuCo experiments make traditional
methods of data distribution impractical. Instead, users will interact with the
data directly where it is stored. The maturation of cloud technologies now
makes this possible.

We will leverage \href{https://www.dandiarchive.org/}{DANDI}, which utilises
Amazon S3 storage, for hosting the data. Additionally, we will provide software
to visualise and analyse data using Amazon EC2 instances, thereby minimising
the need for time-consuming data transfers.

Handling and sharing continuous behavioural and neural recordings of this scale
presents unique challenges. Runtime performance is one of them. If we
encounter unacceptable delays, we will explore advanced optimisation
strategies, such as parallel processing and resource-efficient cloud
configurations.

\subsubsection{Data visualisation}

Our visualisation tools need to display very large datasets at different
temporal scales, from milliseconds to weeks and months, and they need to be web
based.
%
We will use multi-resolution visualisation techniques, which store data at
various resolutions, and use the appropriate resolution for each zoom level.
%
Web-based visualisation will be optimised using web workers.

\subsubsection{Spike sorting}

Spike sorting is specially challenging in NaLoDuCo experimentation since we
want to track individual neurons of freely moving mice for weeks to months.
%
In addition, we need online spike sorting, to allow experiments driven
by real-time machine learning inference, as described below.
%
We will evaluate methods for tracking neurons over long periods of time
(e.g., [\href{https://pubmed.ncbi.nlm.nih.gov/38985568/}{5}]) and for online sorting
(e.g., [\href{https://pubmed.ncbi.nlm.nih.gov/16488479/}{6}]). If needed, we will develop
new methods, as we are experienced on the subject.

\subsubsection{Data analysis}

The very large size of NaLoDuCo experimental data, the fact that the statistics
of these data change across time, and the requirement for real-time and
close-loop inference create new challenges to conventional machine learning
data analysis methods.
%
We will evaluate how existing methods targeting the focus areas described above
cope with these challenges and, if necessary, create new ones.

For behavioural data, we will investigate methods to:

\begin{itemize}

    \item track multiple body parts of animals (e.g.,
        [\href{https://pubmed.ncbi.nlm.nih.gov/30127430/}{7}] and a
        switching-linear-dynamical method using RFIDs that we will develop),

    \item infer kinematics of foraging mice (e.g.,
        [\href{https://github.com/joacorapela/lds\_python}{8},\href{https://www.cambridge.org/core/books/fundamentals-of-object-tracking/A543B0EA12957B353BE4B5D0602EE945}{9}]),

    \item segment behaviour into discrete states (e.g.,
        [\href{https://pubmed.ncbi.nlm.nih.gov/26687221/}{10}]
        and a hierarchical HMM that we will develop),

    \item infer the rules that govern mice behaviour from behavioural
        observations only (i.e., policy inference) (e.g.,
        [\href{https://arxiv.org/abs/2311.13870v2}{11}]).

\end{itemize}

For neural data, we will investigate methods to:

\begin{itemize}

    \item estimate low-dimensional continual representations of neural activity
        (i.e., latents inference) (e.g.,
        [\href{https://papers.nips.cc/paper_files/paper/2011/hash/7143d7fbadfa4693b9eec507d9d37443-Abstract.html}{12}]),

    \item segment neural activity into discrete states (e.g.,
        [\href{https://pubmed.ncbi.nlm.nih.gov/21299424/}{13}]),

    \item decode environment variables from neural activity (e.g.,
        [\href{https://pubmed.ncbi.nlm.nih.gov/25973549/}{14}]).

\end{itemize}

\putbib
\end{bibunit}

\pagebreak
\subsection{Applicant and team capability to deliver}

\begin{instruction}
Word limit: 1,650

Why are you the right individual or team to successfully deliver the proposed
work?

What the assessors are looking for in your response

Please ensure the current job titles of the core team members are included
here to ensure eligibility can be established for the core team roles assigned.
Find out more about
\href{https://www.ukri.org/publications/roles-in-funding-applications/roles-in-funding-applications-eligibility-responsibilities-and-costings-guidance/}{UKRI’s
core team roles in funding applications} and our
\href{https://www.ukri.org/councils/bbsrc/guidance-for-applicants/check-if-youre-eligible-for-funding/applicants-and-co-applicants/}{eligibility
guidance}.

Evidence of how you, and if relevant your team, have:

\begin{itemize}

    \item the relevant experience (appropriate to career stage) to deliver the proposed
work

    \item the right balance of skills and expertise to cover the proposed work

    \item the appropriate leadership and management skills to deliver the work and
your approach to develop others

    \item contributed to developing a positive research environment and wider
community

\end{itemize}

You may demonstrate elements of your responses in visual form if relevant.

Further details are provided in the Funding Service.

The word limit for this section is 1,650 words: 1,150 words to be used for R4RI
modules (including references) and, if necessary, a further 500 words for
Additions.

Use the Résumé for Research and Innovation (R4RI) format to showcase the range
of relevant skills you and, if relevant, your team (project and project
co-leads, researchers, technicians, specialists, partners and so on) have and
how this will help deliver the proposed work. You can include individuals’
specific achievements but only choose past contributions that best evidence
their ability to deliver this work.

Complete this section using the R4RI module headings listed. Use each
heading once and include a response for the whole team, see the UKRI
guidance on R4RI. You should consider how to balance your answer, and
emphasise where appropriate the key skills each team member brings:

\begin{itemize}

    \item contributions to the generation of new ideas, tools, methodologies, or
knowledge

    \item the development of others and maintenance of effective working relationships

    \item contributions to the wider research and innovation community

    \item contributions to broader research or innovation users and audiences and
towards wider societal benefit

\end{itemize}

Additions

Provide any further details relevant to your application. This section is
optional and can be up to 500 words. You should not use it to describe
additional skills, experiences, or outputs, but you can use it to describe any
factors that provide context for the rest of your R4RI (for example, details of
career breaks if you wish to disclose them).

Complete this as a narrative. Do not format it like a CV.

References may be included within this section.

The roles in funding applications policy has descriptions of the different
project roles.

\end{instruction}

% 
\begin{instruction}
Word limit: 1,650

Why are you the right individual or team to successfully deliver the proposed
work?

What the assessors are looking for in your response

Please ensure the current job titles of the core team members are included
here to ensure eligibility can be established for the core team roles assigned.
Find out more about
\href{https://www.ukri.org/publications/roles-in-funding-applications/roles-in-funding-applications-eligibility-responsibilities-and-costings-guidance/}{UKRI’s
core team roles in funding applications} and our
\href{https://www.ukri.org/councils/bbsrc/guidance-for-applicants/check-if-youre-eligible-for-funding/applicants-and-co-applicants/}{eligibility
guidance}.

Evidence of how you, and if relevant your team, have:

\begin{itemize}

    \item the relevant experience (appropriate to career stage) to deliver the proposed
work

    \item the right balance of skills and expertise to cover the proposed work

    \item the appropriate leadership and management skills to deliver the work and
your approach to develop others

    \item contributed to developing a positive research environment and wider
community

\end{itemize}

You may demonstrate elements of your responses in visual form if relevant.

Further details are provided in the Funding Service.

The word limit for this section is 1,650 words: 1,150 words to be used for R4RI
modules (including references) and, if necessary, a further 500 words for
Additions.

Use the Résumé for Research and Innovation (R4RI) format to showcase the range
of relevant skills you and, if relevant, your team (project and project
co-leads, researchers, technicians, specialists, partners and so on) have and
how this will help deliver the proposed work. You can include individuals’
specific achievements but only choose past contributions that best evidence
their ability to deliver this work.

Complete this section using the R4RI module headings listed. Use each
heading once and include a response for the whole team, see the UKRI
guidance on R4RI. You should consider how to balance your answer, and
emphasise where appropriate the key skills each team member brings:

\begin{itemize}

    \item contributions to the generation of new ideas, tools, methodologies, or
knowledge

    \item the development of others and maintenance of effective working relationships

    \item contributions to the wider research and innovation community

    \item contributions to broader research or innovation users and audiences and
towards wider societal benefit

\end{itemize}

Additions

Provide any further details relevant to your application. This section is
optional and can be up to 500 words. You should not use it to describe
additional skills, experiences, or outputs, but you can use it to describe any
factors that provide context for the rest of your R4RI (for example, details of
career breaks if you wish to disclose them).

Complete this as a narrative. Do not format it like a CV.

References may be included within this section.

The roles in funding applications policy has descriptions of the different
project roles.

\end{instruction}


\pagebreak
\subsection{Project partners}

\begin{instruction}

Add details about any project partners’ contributions. If there are no project
partners, you can indicate this on the Funding Service.

A project partner is a collaborating organisation who will have an integral
role in the proposed research. This may include direct (cash) or indirect
(in-kind) contributions such as expertise, staff time or use of facilities.
Project partners may be in industry, academia, third sector or government
organisations in the UK or overseas, including partners based in the EU.

If you are applying via the IPA or LINK scheme, please include details of
industry partners here.

If applying under the BBSRC-NSF lead agency scheme, please include details
of your US partner here.

Add the following project partner details:

\begin{itemize}

    \item the organisation name and address (searchable via a drop-down list or enter
the organisation’s details manually, as applicable)

    \item the project partner contact name and email address

    \item the type of contribution (direct or in-direct) and its monetary value
\end{itemize}

If a detail is entered incorrectly and you have saved the entry, remove the
specific project partner record and re-add it with the correct information.

For audit purposes, UKRI requires formal collaboration agreements to be put in
place if an award is made.

\end{instruction}

% 
\begin{instruction}

Add details about any project partners’ contributions. If there are no project
partners, you can indicate this on the Funding Service.

A project partner is a collaborating organisation who will have an integral
role in the proposed research. This may include direct (cash) or indirect
(in-kind) contributions such as expertise, staff time or use of facilities.
Project partners may be in industry, academia, third sector or government
organisations in the UK or overseas, including partners based in the EU.

If you are applying via the IPA or LINK scheme, please include details of
industry partners here.

If applying under the BBSRC-NSF lead agency scheme, please include details
of your US partner here.

Add the following project partner details:

\begin{itemize}

    \item the organisation name and address (searchable via a drop-down list or enter
the organisation’s details manually, as applicable)

    \item the project partner contact name and email address

    \item the type of contribution (direct or in-direct) and its monetary value
\end{itemize}

If a detail is entered incorrectly and you have saved the entry, remove the
specific project partner record and re-add it with the correct information.

For audit purposes, UKRI requires formal collaboration agreements to be put in
place if an award is made.

\end{instruction}


\pagebreak
\subsection{Project partners: statement of support}

\begin{instruction}

Word limit: 3,000

Only complete a statement of support if you have named project partners in the
project partner section above. A statement is required to be provided from each
partner you named in the ‘Project partners’ section.

If you are applying via the IPA or LINK scheme, please include details of
industry partner support here.

What the assessors are looking for in your response

A project partner is a collaborating organisation who will have an integral
role in
the proposed research. This may include direct (cash) or indirect (in-kind)
contributions such as expertise, staff time or use of facilities.

Each statement should:

\begin{itemize}

    \item confirm the partner’s commitment to the project

    \item clearly explain the value, relevance, and possible benefits of the
    work to them

    \item describe any additional value that they bring to the project

\end{itemize}

Ensure you have prior agreement from project partners so that, if you are
offered funding, they will support your project as indicated in the ‘Project
partners’ section.

For audit purposes, UKRI requires formal collaboration agreements to be put in
place if an award is made.

Do not provide a statement of support from host and project co-leads’ research
organisations.

Do not provide a statement of support from collaborators. Contributions from
collaborators not listed as project partners can be outlined in ‘Applicant and
team capability to deliver’.

\end{instruction}
% \begin{instruction}

Word limit: 3,000

Only complete a statement of support if you have named project partners in the
project partner section above. A statement is required to be provided from each
partner you named in the ‘Project partners’ section.

If you are applying via the IPA or LINK scheme, please include details of
industry partner support here.

What the assessors are looking for in your response

A project partner is a collaborating organisation who will have an integral
role in
the proposed research. This may include direct (cash) or indirect (in-kind)
contributions such as expertise, staff time or use of facilities.

Each statement should:

\begin{itemize}

    \item confirm the partner’s commitment to the project

    \item clearly explain the value, relevance, and possible benefits of the
    work to them

    \item describe any additional value that they bring to the project

\end{itemize}

Ensure you have prior agreement from project partners so that, if you are
offered funding, they will support your project as indicated in the ‘Project
partners’ section.

For audit purposes, UKRI requires formal collaboration agreements to be put in
place if an award is made.

Do not provide a statement of support from host and project co-leads’ research
organisations.

Do not provide a statement of support from collaborators. Contributions from
collaborators not listed as project partners can be outlined in ‘Applicant and
team capability to deliver’.

\end{instruction}


\pagebreak
\subsection{Trusted research and innovation (TR\&I)}

\begin{instruction}

Word limit: 100

Does the proposed work involve international collaboration in a sensitive
research or technology area?

What the assessors are looking for in your response

Demonstrate how your proposed international collaboration relates to TR\&I,
including:

\begin{itemize}

	\item list the countries your international project co-leads, project
partners and visiting researchers, or other collaborators are based in

	\item if international collaboration is involved, explain whether this project is relevant to one or more of the \href{https://www.gov.uk/government/publications/national-security-and-investment-act-guidance-on-notifiable-acquisitions/national-security-and-investment-act-guidance-on-notifiable-acquisitions}{17 areas} of the UK National Security
and Investment (NSI) Act

	\item if one or more of the 17 areas of the UK National Security and
Investment (NSI) Act are involved list the areas

\end{itemize}

If your proposed work does not involve international collaboration, you will be
able to indicate this in the Funding Service. If your proposed work does not
involve international collaboration, please indicate this by stating ‘N/A’.
We may contact you following submission of your application to provide
additional information about how your proposed project will comply with our
approach and expectation towards TR\&I, identifying potential risks and the
relevant controls you will put in place to help manage these risks.

\end{instruction}

% \input{trustedResearchAndInnovation}

\pagebreak
\subsection{Resources and cost justification}

\begin{instruction}

Word limit: 1,000

What will you need to deliver your proposed work and how much will it cost?

What the assessors are looking for in your response

The FEC of your project can be up to a maximum of £2 million. We will fund
80% of the FEC. For example, if the FEC cost of your project is equal £2
million, we will fund £1.6 million and your research organisation will be
expected to fund £400,000.

Please note, equipment over £10,000 is funded by BBSRC at 50\%. The
Funding Service does not currently have the ability to record this. For this round
we ask that you include equipment over £10,000 in ‘Exceptions’ at 100\% of
cost. We will cut this to 50\% at award. You must ensure you have prior
agreement from your research organisation to fund the remaining 50\%.

Justify the application’s more costly resources, in particular:

\begin{itemize}

	\item project staffsignificant travel for field work or collaboration (but
not regular travel between collaborating organisations or to conferences)

	\item any equipment that will cost more than £10,000

	\item any consumables beyond typical requirements, or that are required in
exceptional quantities

	\item all facilities and infrastructure costs

	\item all resources that have been costed as ‘Exceptions’

\end{itemize}

Assessors are not looking for detailed costs or a line-by-line breakdown of all
project resources. Overall, they want you to demonstrate how the resources
you anticipate needing for your proposed work:

\begin{itemize}

	\item are comprehensive, appropriate, and justified

	\item represent the optimal use of resources to achieve the intended
outcomes

	\item maximise potential outcomes and impacts

\end{itemize}

\end{instruction}

% 
UK Resources:

Total cost estimate: £1,485,198.15

Research council contribution: £1,188,158.52

2 x 0.1 FTE PI, 1 x 0.5 FTE PDRA, 2 x 1.0 FTE RSE

\vspace{0.1in}
US Resources:

Total cost estimate: \$700.000

1 x 0.5 FTE scientist 1

\vspace{0.1in}
Total funder contribution estimate:

£1,741,158.52 (£1,188,158.52 + £553,000 (\$700.000 at exchange rate 0.79))



\pagebreak
\subsection{Data management and sharing}

\begin{instruction}

Word limit: 500

How will you manage and share data collected or acquired through the
proposed research?

What the assessors are looking for in your response
Provide a data management plan that clearly details how you will comply with
UKRI’s published data sharing policy, which includes detailed guidance notes.

\end{instruction}

% \input{dataManagement}

\pagebreak
\subsection{Facilities}

\begin{instruction}

Word limit: 250
Does your proposed research require the support and use of a facility?

What the assessors are looking for in your response

If you will need to use a facility, follow your proposed facility’s normal access
request procedures. Ensure you have prior agreement so that if you are offered
funding, they will support the use of their facility on your project.

For each requested facility you will need to provide the:

\begin{itemize}

	\item name of facility, copied and pasted from the \href{https://ukri-tfs-prod-assets.s3.eu-west-2.amazonaws.com/Facility+Information+for+TFS+updated+Sept+24.docx}{facility information list (DOCX, 42KB)}

	\item proposed usage or costs, or costs per unit where indicated on the facility
information list

	\item confirmation you have their agreement where required

\end{itemize}

Facilities should only be named if they are on the facility information list above.
If you will not need to use a facility, you will be able to indicate this in the
Funding Service.

\end{instruction}

\pagebreak
\subsection{Ethics and responsible research and innovation (RRI)}

\begin{instruction}

Word limit: 500

What are the ethical or RRI implications and issues relating to the proposed
work? If you do not think that the proposed work raises any ethical or RRI
issues, explain why.

What the assessors are looking for in your response

Demonstrate that you have identified and evaluated:

\begin{itemize}

	\item the relevant ethical or responsible research and innovation considerations

	\item how you will manage these considerations

\end{itemize}

If you are collecting or using data you should identify:

\begin{itemize}

	\item any legal and ethical considerations of collecting, releasing or
storing the data (including consent, confidentiality, anonymisation, security
and other ethical considerations and, in particular, strategies to not preclude
further reuse of data)

	\item formal information standards that your proposed work will comply with

\end{itemize}

You may demonstrate elements of your responses in visual form if relevant.
Further details are provided in the Funding Service.

\end{instruction}

\pagebreak
\subsection{Genetic and biological risk}

\begin{instruction}

Word limit: 700

Does your proposed research involve any genetic or biological risk?

What the assessors are looking for in your response

In respect of animals, plants or microbes, are you proposing to:

\begin{itemize}

	\item use genetic modification as an experimental tool, like studying gene
function in a genetically modified organism

	\item release genetically modified organisms

	\item ultimately develop commercial and industrial genetically modified
outcomes

\end{itemize}

If yes, provide the name of any required approving body and state if approval
is already in place. If it is not, provide an indicative timeframe for
obtaining the required approval.Identify the organism or organisms as a plant,
animal or microbe and specify the species and which of the three categories the
research relates to.

Identify the genetic and biological risks resulting from the proposed research,
their implications, and any mitigation you plan on taking. Assessors will want
to know you have considered the risks and their implications to justify that
any identified risks do not outweigh any benefits of the proposed research.

If this does not apply to your proposed work, you will be able to indicate this
in the Funding Service.

\end{instruction}

\pagebreak
\subsection{Research involving the use of animals}

\begin{instruction}

Does your proposed research involve the use of vertebrate animals or other
organisms covered by the Animals Scientific Procedures Act?

What the assessors are looking for in your response

If you are proposing research that requires using animals, download and
complete the Animals Scientific Procedures Act template (DOCX, 74KB),
which contains all the questions relating to research using vertebrate animals or
other Animals (Scientific Procedures) Act 1986 regulated organisms.

Save it as a PDF, ensuring it is no larger than 8MB. The Funding Service will
provide document upload details when you apply. If this does not apply to your
proposed work, you will be able to indicate this in the Funding Service.

\end{instruction}

\pagebreak
\subsection{Conducting research with animals overseas}

\begin{instruction}

Word limit: 700

Will any of the proposed animal research be conducted overseas?

What the assessors are looking for in your response

If you are proposing to conduct overseas research, it must be conducted in
accordance with welfare standards consistent with those in the UK, as in
Responsibility in the use of animals in bioscience research. Ensure all
named applicants in the UK and overseas are aware of this requirement.
Ensure all named applicants in the UK and overseas are aware of this
requirement.

If your application proposes animal research to be conducted overseas, you
must provide a statement in the text box. Depending on the species involved,
you may also need to upload a completed template for each species listed.

Statement

Provide a statement to confirm that:

\begin{itemize}

	\item all named applicants are aware of the requirements and have agreed to
abide by them

	\item this overseas research will be conducted in accordance with welfare
standards consistent with the principles of UK legislation

	\item the expectation set out in Responsibility in the use of animals in bioscience
research will be applied and maintained

	\item appropriate national and institutional approvals are in place

\end{itemize}

Templates

Overseas studies proposing to use non-human primates, cats, dogs, equines or
pigs will be assessed during NC3Rs review of research applications. Provide
the required information by completing the template from the question
‘Research involving the use of animals’.

For studies involving other species, such as:

\begin{itemize}

	\item rodents

	\item rabbits

	\item sheep

	\item goats

	\item pigs

	\item cattle

	\item xenopus laevis and xenopus tropicalis

	\item zebrafish

\end{itemize}

Select, download, and complete the relevant Word checklist or checklists by
exploring NC3Rs checklist for the use of animals overseas.

Save your completed template as a PDF and upload to the Funding Service. If
you use more than one checklist template, save it as a single PDF.
The Funding Service will provide document upload details when you apply.

If conducting research with animals overseas does not apply to your proposed
work, you will be able to indicate this in the Funding Service.

\end{instruction}

\pagebreak
\subsection{Research involving human participation}

\begin{instruction}

Word limit: 700

Will the project involve the use of human subjects or their personal information?
What the assessors are looking for in your response

If you are proposing research that requires the involvement of human subjects,
provide the name of any required approving body and whether approval is
already in place.Justify the number and the diversity of the participants involved, as well as any
procedures.

Provide details of any areas of substantial or moderate severity of impact.

If this does not apply to your proposed work, you will be able to indicate this in
the Funding Service.

\end{instruction}

\pagebreak
\subsection{Research involving human tissues or biological samples}

\begin{instruction}

Word limit: 700

Does your proposed research involve the use of human tissues, or biological
samples?

What the assessors are looking for in your response

If you are proposing work that involves human tissues or biological samples,
provide the name of any required approving body and whether approval is
already in place.

Justify the use of human tissue or biological samples specifying the nature and
quantity of the material to be used and its source.

If this does not apply to your proposed work, you will be able to indicate this in
the Funding Service.

\end{instruction}

\end{document}
