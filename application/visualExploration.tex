
Visualizations are essential  for extracting meaning from any dataset.

Because NaLoDuCo datasets are so large, it is not practical to download them,
and visualisation methods should run where the data lives, on the cloud or on
institutional clusters.

We will develop visualisation functionality for continuous
(Section~\ref{sec:visContinuous}) and for epoched
datasets, where the epochs could be built around events derived by advanced
machine learning methods 
(Section~\ref{sec:visEpoched}).

\subsubsubsection{Continuous visualisations}
\label{sec:continousVis}

With continuous visualisations users will be able to seamlessly explore large
behaviorald and neural datasets, expanding weeks to months. They should easily be able to zoom out and
explore the data at the month level and then zoom in to visualize the data at
the millisecond resolution. We aim at providing an experience similar to that
obtained from Google Maps, where we can zoom back and forth from a whole world
view to a single house view.

To make this possible we will use a combination of tiling, hierarchical
storage, and streaming techniques.

\paragraph{Multi-Resolution Tiling.}

- Large volumetric datasets will be preprocessed into small tiles at multiple resolutions.
- When a user zooms in, only the necessary tiles at the appropriate resolution
will be loaded, reducing bandwidth and memory requirements.

\paragraph{Cloud and Distributed Storage}

- Data will be stored in cloud object storage (e.g., AWS S3 used by DANDI) and
networked file systems (e.g., ceph used by institutional clusters)

- This allows seamless access to massive datasets without requiring local storage.

\paragraph{Efficient Streaming and On-Demand Loading}

- Instead of loading entire datasets into memory, the system will stream to
cliens only the visible portion of the data.

- This is similar to how Google Maps loads small sections of a map as you navigate.

\paragraph{WebGL-Based Rendering}

- We will use WebGL for GPU-accelerated rendering, allowing real-time visualization of high-resolution images directly in a web browser.

\paragraph{Sparse Annotation Handling}

- Annotations (e.g., sorted unit) will be stored in separate, compressed layers and fetched as needed.

- This will reduce memory usage and speeds up interactions.

\paragraph{Open Source \& API Integration}
- The system will be open-source, meaning researchers can integrate it with their own pipelines.
- It will support Python APIs for automated data handling and visualization.

\paragraph{Related software}

- Google maps

- Webknossos

\subsubsubsection{Intelligently epoched visualisations}
\label{sec:intelligentlyEpochedVis}

We will provide users means to visualise behavioral and neural datasets epoched
around events of interest. For example, users will be able to visualize spikes
counts of a given mouse, when the mouse was at a particular odor delivery port
of the home cage, in a time range of the day (e.g., 9pm-11pm) and when its
neural activity was in given neural state (as inferred from a hidden Markov
model). That is, the visualisation indices will be built with spatio-temporal
information and with information obtained by advanced data analysis methods.

There will be a close loop between visualisations and data analysis. As
described above, data analysis will shape visualisations. But also
visualisations will demand new data analysis, that in turn will in turn
generate new visualisations.

Enabling these visualisation will require advanced indexing systems in
high-performance databases, where each
data item to be visualised (e.g., spike times, local field potential, subject
position, velocity and acceleration) can be associated with a dynamically
evolving set of features.

\subsubsubsection{Related work}

- a spatio-temporal indexing system is part of the current AEON platform

- the AEON platform uses MySQL databases

\subsubsubsection{Outputs}

\begin{enumerate}

    \item visualisations for continuous behavioural and neural recording

    \item visualisations for epoched behavioural and neural recording

    \item visualisations for model outputs

    \item indexing system to support intelligent visualisations

    \item deployment of the above items to allow users to visualise NaLoDuCo
        DANDI datasets on the cloud

\end{enumerate}
