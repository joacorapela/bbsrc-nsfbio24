\subsubsection{Context}

Conventional systems neuroscience experiments are typically short in duration
and often place significant constraints on subjects behaviours to simplify data
analysis.
%
However, these restrictions may limit our ability to observe critical
aspects of brain function and behaviour that only manifest in more naturalistic
and extended conditions.

At the Sainsbury Wellcome Centre (SWC) and Gatsby Computational Neuroscience
Unit (GCNU) we are pioneering Naturalistic, Long-Duration, and Continual
(NaLoDuCo) foraging experiments in mice that span weeks to months. During these
experiments, we collect high-resolution behavioural and neural recordings in
naturalistic settings.

This novel  approach will enable researchers to explore neural mechanisms
underlying ethological behaviours in naturalistic environments over months, for
the first time.  The experiments will shed new light on a wide range of poorly
understood neural mechanisms, including how the brain structures complex
behavioural sequences as a function of the animal needs, learning and social
dynamics.
%
The data generated from NaLoDuCo experiments represent an entirely new resource
in neuroscience, with the potential to drive breakthroughs and discoveries that
are beyond the reach of traditional experiments.

While experiments in neuroscience that are naturalistic, long-duration, or
continuous have been conducted in the past (e.g.,
[\href{https://pubmed.ncbi.nlm.nih.gov/37656619/}{1}]), to the best of our
knowledge, we are the first to integrate all three of these features in a
single experimental paradigm.
%
Experiments of this type have been advocated by experts in the field years ago
([\href{https://pubmed.ncbi.nlm.nih.gov/31600508/}{2}], p.19), yet they have
not been implemented so far.

We anticipate that this new type of experimentation will become mainstream in
the coming years.
%
However, experiments spanning weeks to months generate extremely large
datasets—often reaching hundreds of terabytes—which present substantial
challenges across data acquisition, management, distribution, visualization,
and analysis.
%
Together, with our US partner, the Allen Institute for Neural Dynamics (AIND),
we will address these challenges building software infrastructure to help
scientists around the world perform NaLoDuCo experiments.

% Since the project started in 2021, our UK business partner, NeuroGEARS Ltd.\
% has been contracted by the SWC to lead the implementation of the NaLoDuCo
% experimental framework. It also provides services to the AIND.

\subsubsection{Focus areas}

Developing platform technologies for:

\begin{description}

    \item[Experimental Control, Data Acquisition \& Management] Controlling
        sophisticated experiments and efficiently gathering and organising
        massive datasets over extended periods.

    \item[Data Sharing] Providing global access to large-scale datasets.

    \item[Data Visualisation] Building web-based visualisations for very large
        behavioural and neural data.

    \item[Data Analysis] Characterising behavioural and neural recordings with
        advanced machine learning methods.

\end{description}

\subsubsection{Synergistic developments}

Our team is highly qualified to deliver this proposal, with world-class
expertise in experimental and computational neuroscience, as well as machine
learning.
%
Both partners also include talented research software engineers, creating a
solid foundation for achieving the project goals.

% The main goal of the AIND is to understand how the brain works at the level of individual
% neurons.
%
% Central to its mission is the development of neuro-technologies to acquire and
% distribute massive ammounts of neural data.

The AIND is also investigating foraging behavior, but using head-fixed mice.
They do not probe freely moving and naturalistic behaviour, but are able to
perform electrophysiological recordings more densely than the SWC.
%
The experimental approaches to foraging by the AIND and SWC are complementary
and this collaboration will greatly benefit both of them.

Currently, both GCNU and AIND are independently developing methods to address
the previous focus areas. We will join forces to co-develop these methods and
our foraging research programs, leveraging our combined expertise for greater
impact.
