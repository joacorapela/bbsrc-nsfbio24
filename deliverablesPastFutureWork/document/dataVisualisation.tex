
\subsection{Need for cloud-based visualisation on the cloud}

A unique feature of the NaLoDuCo recordings collected at the SWC and the AIND,
is that they are long-duration and continual. Greatest insights will come from
investigating these recordings as a whole, and not by analysing separately its
parts. For example, the analysis of shorter duration recordings will not be
able to capture long-term temporal dependencies in neural activity, that could
be critical to understand infradian modulations of behaviour. Thus, we need
software infrastructure to browse and visualise week- to month-long
experimental recordings on the order of hundreds of terabytes. It is not
feasible to download these huge datasets in order to visualise them. Hence,
\textbf{offline data visualisation needs to be done on the cloud}, as in Neurosift.

\subsection{Deliverables}

\begin{enumerate}

    \item web-based dashboard for \textbf{online} experiment monitoring.

    \item web-based dashboard for \textbf{online} data analysis and
    visualisation of its results.

    \item web-based dashboard for \textbf{offline} visualisation of NaLoDuCo
    behavioural and neural recordings on DANDI.

    These visualizations should allow users to efficiently explore very large
    behavioral, neural and data analysis datasets. For instance, users should
    be able to quickly visualise videos of mice when they were next to any
    location of the arena, at any time of the day, and in any behavioral state
    (as inferred by a Hidden Markov Model). These efficient data explorations
    will require sophisticated data indexing schemes.

    \item web-based dashboard for \textbf{offline} visualisation of data analysis
    results on DANDI.

\end{enumerate}

\subsection{Questions}

\begin{itemize}

    \item can I see the visualisation tools from the AIND?

    \item does the AIND has visualisation tools running on the cloud?

    \item what visualisation tools do we have at the SWC? developed by
    Datajoint?

    \item is the AIND collaborating with Jeremy Magland?

\end{itemize}

\subsection{Previous work}

\begin{itemize}

    \item Neurosift
        (\href{https://github.com/flatironinstitute/neurosift}{repo},
        \href{https://joss.theoj.org/papers/10.21105/joss.06590}{paper}) allows to
    visualise shorter-duration behavioural and neural recordings in DANDI.

    \item Dendro (\href{https://github.com/magland/dendro}{repo}) allows to
    perform analysis on the cloud and visualise the results of such analysis

    \item the SWC has developed visualisations in Bonsai for experimental
    monitoring.

    \item offline and precomputed visualisations developed at the SWC, of some
    experimental variables (e.g., total distance that animals have moved the
    wheel, proportion of time animals spent on the nest, corridor or food
    patches).

    \item offline and precomputed visualisations developed at the AIND for
    shorter-duration experiments.

    \item offline and precomputed visualisations from IBL for short-duration
    experiments.

\end{itemize}

\subsection{Future work}

\begin{itemize}

    \item create web-based dashboard to monitor NaLoDuCo experiments.

    \item create web-based dashboard for \textbf{online} data analysis and
    visualisation of its results.

    \item create web-based dashboard for \textbf{offline} visualisation of NaLoDuCo
    behavioural and neural recordings on DANDI.

    Neurosift has been designed to visualise relatively short duration
    datasets. We will extend it with data pyramids (e.g.,
    \url{https://github.com/carbonplan/ndpyramid}) to enable it to operate on
    long-duration recordings.

    \item web-based dashboard for \textbf{offline} visualisation of data analysis
    results on DANDI.

    Dendro has been used to analyze on the cloud relatively small datasets. We
    will extend Dendro to analyze very large datasets and visualize the results
    of these analysis.

\end{itemize}

