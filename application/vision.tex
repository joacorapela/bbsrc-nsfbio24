\subsubsection{Context}

Conventional systems neuroscience experiments are typically short in duration
and often place significant constraints on subjects behaviours to simplify data
analysis.
%
However, these restrictions may limit our ability to observe critical
aspects of brain function and behaviour that only manifest in more naturalistic
and extended conditions.

At the Sainsbury Wellcome Centre (SWC) and Gatsby Computational Neuroscience
Unit (GCNU) we are pioneering Naturalistic, Long-Duration, and Continual
(NaLoDuCo) foraging experiments in mice that span weeks to months. During these
experiments, we collect high-resolution behavioural and neural recordings in
naturalistic settings.

This novel experimental approach will enable researchers to explore neural
mechanisms underlying naturalistic behaviour over extended periods for the first
time, offering the possibility of uncovering insights across a wide range of
phenomena, including long-term behavioural adaptation, neural plasticity, and
learning.
%
The data generated from NaLoDuCo experiments represent an entirely new resource
in neuroscience, with the potential to drive breakthroughs and discoveries that
are beyond the reach of traditional experiments.

While experiments in neuroscience that are naturalistic, long-duration, or
continuous have been conducted in the past
(e.g., [\href{https://pubmed.ncbi.nlm.nih.gov/37656619/}{1}]), to the best of our
knowledge, we are the first to integrate all three of these features in a
single experimental paradigm.

Our US collaborator, the Allen Institute for Neural Dynamics (AIND) is also investigating
foraging, but using head-fixed mice. Key to their mission is distributing very large Neuroscience datasets,
and providing functionality to process them on the cloud.

Since the project started in 2021, our UK business partner, NeuroGEARS Ltd.\ 
has been contracted by the SWC to lead the implementation of the NaLoDuCo
experimental framework. It also provides services to the AIND.

The extremely large datasets--on the order of hundreds of terabytes--gathered
from experiments spanning weeks to months pose significant challenges in data
acquisition, visualisation, and analysis.
%
Together, the GCNU, SWC, AIND and NeuroGEARS will address these challenges,
co-develop this new type of experimentation, share expertise and build software
infrastructure to help scientists around the world perform NaLoDuCo
experiments.

\subsubsection{Focus areas}

The focus areas of the proposed project are:

\begin{description}

    \item[Data Collection \& Management] Efficiently gathering and organising
        massive datasets over extended periods.

    \item[Data Sharing] Providing global access to large-scale datasets.

    \item[Data Visualisation] Developing efficient web-based tools to visualise
        very large behavioural and neural datasets.

    \item[Data Analysis] Characterising behavioural and neural recordings.

\end{description}

\subsubsection{Cross fertilisation}

The foraging experiments at the AIND are different from those at the SWC. They
do not probe freely moving and naturalistic behaviour, but are able to perform
electrophysiological recordings more densely than those at the SWC.
%
These experimental approaches to foraging are complementary and this
collaboration will greatly benefit both of them.

Currently, both GCNU and AIND are independently developing methods to address
the previous focus areas. We will join forces to co-develop these areas and
our foraging research programs, leveraging our combined expertise for greater
impact.
