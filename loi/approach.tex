\subsubsection{Foundational progress: experimental control, data acquisition \& management}

We have developed an innovative platform for housing of mice in large arenas
(\textgreater 2m diameter) enabling precise behavioural manipulation and
high-resolution monitoring
(\href{https://www.gatsby.ucl.ac.uk/~rapela/bbsrc\_nsfbio/figures/foragingArena.png}{online
figure},
[\href{https://www.abstractsonline.com/pp8/?_gl=1*it0gi6*_gcl_au*MTUyNDE0NDQwLjE3Mjc2OTgyODM.*_ga*MTUxNDI2NDg5LjE3Mjc2OTgyODM.*_ga_T09K3Q2WDN*MTcyOTUwNDUzNy4yLjEuMTcyOTUwNDY3Ny41NC4wLjA.#!/20433/presentation/22271}{2}]).
%
We have openly shared software for supporting data
acquisition~[\href{https://github.com/SainsburyWellcomeCentre/aeon_acquisition}{3}]
and
management~[\href{https://github.com/SainsburyWellcomeCentre/aeon_mecha}{4}] in this
arena.
%
Additionally, the platform supports continuous, long term monitoring of neural
activity with Neuropixels probes.
%
This setup has allowed us to collect several week-long datasets with single and
multiple mice per arena.


\subsubsection{Intelligent experimental control}

Bonsai is an unparalleled software for experimental control that both the SWC
and the AIND are using to control their foraging experiments.
%
We are currently adding machine learning functionality to Bonsai, funded by
BBSRC
[\href{https://gow.bbsrc.ukri.org/grants/AwardDetails.aspx?FundingReference=BB\%2FW019132\%2F1}{5}].

We will continue enhancing Bonsai with machine learning methods developed in
this project.
%
Our goal is to enable a new type of experimentation, driven by sophisticated
inferences from behavioural and neural recordings. This innovative approach
could be transformative for foraging research.

% To facilitate the replication of our experimental setup by other groups, we
% will share instructions for building foraging arenas, as well as specifications
% of hardware used in them,
%
% and we will improve the documentation of the software repositories for data
% acquisition and management.

\subsubsection{Data sharing}

The very large datasets produced by NaLoDuCo experiments make traditional
methods of data distribution impractical. Instead, users will interact with the
data directly where it is stored. The maturation of cloud technologies now
makes this possible.

We will leverage DANDI [\href{https://www.dandiarchive.org/}{6}], which utilises
Amazon S3 storage, for hosting data. Additionally, we will provide software
to visualise and analyse data using Amazon EC2 instances, thereby minimising
the need for time-consuming data transfers.

Handling and sharing continuous behavioural and neural recordings of this scale
presents unique challenges. Runtime performance is one of them. If we
encounter unacceptable delays, we will explore advanced optimisation
strategies, such as parallel processing and resource-efficient cloud
configurations.

\subsubsection{Data visualisation}

Our visualization tools must efficiently display very large datasets across a
range of temporal scales, from milliseconds to weeks and months, in a web-based
format, as the data will be stored online. To accomplish this, we will
implement multi-resolution visualization techniques that store data at various
resolutions, dynamically adjusting to the appropriate resolution based on zoom
level. Additionally, we will optimize web-based visualizations using web
workers to improve performance and responsiveness.

% \subsubsection{Spike sorting}

% Spike sorting is specially challenging in NaLoDuCo experimentation since we
% want to track individual neurons of freely moving mice for weeks to months.
%
% In addition, we need online spike sorting, to allow experiments driven
% by real-time machine learning inference, as described below.
%
% We will evaluate methods for tracking neurons over long periods of time
% (e.g., [\href{https://pubmed.ncbi.nlm.nih.gov/38985568/}{5}]) and for online sorting
% (e.g., [\href{https://pubmed.ncbi.nlm.nih.gov/16488479/}{6}]). If needed, we will develop
% new methods, as we are experienced on the subject.

\subsubsection{Data analysis}

The very large size of NaLoDuCo experimental data, the fact that the statistics
of these data change across time, and the requirement for real-time and
close-loop inference create new challenges to conventional machine learning
data analysis methods.
%
We will evaluate how existing methods targeting the focus areas cope with these
challenges and, if necessary, create new ones.

For behavioural data, we will investigate methods to:

\begin{description}

    \item[track multiple body parts of animals] (e.g.,
        [\href{https://pubmed.ncbi.nlm.nih.gov/30127430/}{7}] and a
        switching-linear-dynamical method using RFIDs that we will develop),

    \item[infer kinematics of foraging mice] (e.g.,
        [\href{https://github.com/joacorapela/lds\_python}{8},\href{https://www.cambridge.org/core/books/fundamentals-of-object-tracking/A543B0EA12957B353BE4B5D0602EE945}{9}]),

    \item[segment behaviour into discrete states] (e.g.,
        [\href{https://pubmed.ncbi.nlm.nih.gov/26687221/}{10}]
        and a hierarchical HMM that we will develop),

    \item[infer the rules that govern mice behaviour from behavioural
        observations only] (i.e., policy inference) (e.g.,
        [\href{https://arxiv.org/abs/2311.13870v2}{11}]).

\end{description}

For neural data, we will investigate methods to:

\begin{description}

    \item[estimate low-dimensional continual representations of neural
        activity]
        (i.e., latents inference) (e.g.,
        [\href{https://papers.nips.cc/paper_files/paper/2011/hash/7143d7fbadfa4693b9eec507d9d37443-Abstract.html}{12}]),

    \item[segment neural activity into discrete states] (e.g.,
        [\href{https://pubmed.ncbi.nlm.nih.gov/21299424/}{13}]),

    \item[decode environment variables from neural activity] (e.g.,
        [\href{https://pubmed.ncbi.nlm.nih.gov/25973549/}{14}]).

\end{description}
