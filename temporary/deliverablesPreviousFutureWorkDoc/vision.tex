
Conventional systems neuroscience experiments are typically short in duration
and often place significant constraints on subjects behaviours to simplify data
analysis.
%
However, these restrictions may limit our ability to observe critical
aspects of brain function and behaviour that only manifest in more naturalistic
and extended conditions.

At the Sainsbury Wellcome Centre for Neural Circus and Behaviours (SWC) and at
the Gatsby Computational Neuroscience Unit (GCNU) we are performing a radically
new type of experimentation, that is Naturalistic, Long Duration, and Continual
(NaLoDuCo). We are recording behaviourally and electrophysiological data
continuously for weeks to months, while mice forage, individually or socially, in
large naturalistic arenas.

The Allen Institute for Neural Dynamics (AIND) is creating software and hardware
technology to study foraging along two parallel paths. First, they are
investigating the neural basis of foraging behaviour in shorter virtual reality
experiments on head fixed mice. Second, they are studying how freely moving
mice learn over extended periods of time, while they spontaneously explore odour
sources.

The SWC/GCNU and the AIND have complementary expertise in foraging research.
%
Since 2021 the SWC/GCNU have been creating software and hardware infrastructure
to enable NaLoDuCo foraging research, and have successfully used this
technology to perform very long-duration behavioural and neural recordings in
mice foraging in large arenas.
%
Also, the GCNU is a leader in developing advanced neural data analysis methods,
that will be essential to understand data generated by NaLoDuCo experiments.
%
The AIND is performing foraging research, in head-fixed mice using virtual
reality in hours-long experiments, and in freely moving mice spontaneously
exploring odour sources in long-duration experiments.
%
In addition, the AIND is a pioneer in generating and openly disseminating
high-quality behavioural and neural datasets at scale.

Here we propose to join forces to co-develop critical NaLoDuCo experimentation
technology. The cross-fertilisation that this collaboration will generate will
substantially improve data processing practices at the SWC/GCNU and at the
AIND, and lead to a superior NaLoDuCo experimentation technology. We will
thoroughly test this technology in our three experimental setups (1.\ freely
moving foraging at the SWC/GCNU, 2.\ head-fixed foraging at the AIND, and 3.\
freely moving odour exploration at the AIND), and disseminate it openly to
facilitate the adoption of NaLoDuCo experimentation by research groups around
the world.
