\subsubsection{Context}

Conventional systems neuroscience experiments are typically short in duration
and often place significant constraints on subjects behaviours to simplify data
analysis.
%
However, these restrictions may limit our ability to observe critical
aspects of brain function and behaviour that only manifest in more naturalistic
and extended conditions.

At the Sainsbury Wellcome Centre (SWC) and Gatsby Computational Neuroscience
Unit (GCNU) we are pioneering Naturalistic, Long-Duration, and Continual
(NaLoDuCo) foraging experiments in mice that span weeks to months. During these
experiments, we collect high-resolution behavioural and neural recordings in
naturalistic settings.

This novel  approach will enable researchers to explore neural mechanisms
underlying ethological behaviours in naturalistic environments over months, for
the first time.  The experiments will shed new light on a wide range of poorly
understood neural mechanisms, including how the brain structures complex
behavioural sequences as a function of the animal needs, learning, adaptation,
sleep-dependent memory consolidation and social dynamics.
%
The data generated from NaLoDuCo experiments represent an entirely new resource
in neuroscience, with the potential to drive breakthroughs and discoveries that
are beyond the reach of traditional experiments.

While naturalistic, long-duration, or continuous neuroscience experiments have
been conducted in the past
\citep{nagyEtAl23,hoEtAl23,rayEtAl25,weissbrodEtAl13,dhawaleEtAl17}, to the
best of our knowledge, we are the first to integrate all three of these
features in a single experimental paradigm.
%
Experiments of this type have been advocated by experts in the field years ago
\citep[][p19]{dattaEtAl19}, yet they have not been implemented so far.

This new type of experimentation will become mainstream in the coming years.
%
However, experiments spanning weeks to months generate extremely large
datasets—often reaching hundreds of terabytes—which present substantial
challenges across data acquisition, management, distribution, visualisation,
and analysis.
%
Together, with our US partner, the Allen Institute for Neural Dynamics (AIND),
we will address these challenges building software infrastructure to help
scientists around the world perform NaLoDuCo experiments.

Since the project started in 2021, our UK business partner, NeuroGEARS Ltd.\
has been contracted by the SWC to lead the implementation of the NaLoDuCo
experimental framework. It also provides services to the AIND.

\subsubsection{Focus areas}

Developing \textbf{platform technologies} to enable:


\begin{description}

    \item[Scalable Data Sharing] – Establishing infrastructure for \textbf{efficient storage and sharing} of massive, long-duration behavioral and neural datasets across research communities.

	\item[Interactive Data Exploration] – Creating high-performance visualization tools to navigate and analyze multi-scale behavioral and neural recordings, enabling \textbf{immediate} insights into complex datasets.

	\item[Advanced Data Analysis] – Developing \textbf{online and distributed} machine learning algorithms to characterize \textbf{non-stationary} behavioral and neural dynamics in continuous, high-dimensional data streams.

	\item[Scalable Spike Sorting] – Designing \textbf{robust and adaptive spike sorting methods} to accurately assign spikes to neurons in \textbf{long-duration, non-stationary recordings}, supporting both \textbf{real-time and offline analysis}.

	\item[Adaptive Neural Perturbations] – Implementing \textbf{real-time neuromodulatory interventions} to test causal relationships between neural activity and behavior over \textbf{extended time scales}.

\end{description}

\subsubsection{Synergistic developments}

Currently, both GCNU and AIND are independently developing methods to address
the previous focus areas. We will join forces to co-develop these methods and
our foraging research programs, leveraging our combined expertise for greater
impact.

\begin{comment}
\subsubsection{Use case}

A user connects to a dashboard to examine recordings from a one month long
foraging experiments in micce. She begins by \textbf{visualising continuous
behavioural measurements}, such as the kinematics (speed and acceleration) of a
mouse during a months-long experiment measured by the IMU of an ONIX probe.

Next, she \textbf{examines the results of a machine learning analysis},
such as behavioural states inferred by a Switching Hidden Markov Model (SHMM)
using kinematic data.

The SHMM was initially trained on the first two hours of the experiment
and was periodically retrained to \textbf{adapt to non-stationarities},
such as sensor fluctuations, changes in motivation, fatigue, or learning.

She then \textbf{visualises epoched data}, such as SHMM states aligned to
key events—e.g., the onset of a foraging bout in the richer patch at a
specific time of day.

Curious about the neural basis of these SHMM states, she checks the neural
recordings but realises they have not yet been spike-sorted. She runs our
\textbf{offline spike sorting method}, developed for very long continuous
recordings, and performs quality control on its results.

Based on quality metrics, she detects drift in the recorded signal,
adjusts the drift correction parameters, and reruns the sorting algorithm.

		After examining the sorted spikes from a large neural population,  she
finds it challenging to interpret activity across so many neurons.  So she
decides to summarise the population spiking activity by \textbf{estimating
continuous latent variables}.

She then returns to the behavioural visualisation software, integrates
machine learning indices corresponding to these latent variables, and
\textbf{visualises behavioural data aligned to the newly estimated latents}.

Through these explorations, she hypothesises that a peak in a neural latent
variable from the prefrontal cortex signals the moment when mice decide
to begin a foraging bout.

To test this, she \textbf{runs an online machine learning model} to estimate
latent variables from prefrontal cortex activity, predicting when this
peak will occur. She then \textbf{optogenetically inactivates the neural
population} at the forecasted time.

Because inactivation prevented the mouse from initiating a foraging bout,
her hypothesis was supported.
\end{comment}

