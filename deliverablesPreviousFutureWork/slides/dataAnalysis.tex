
\begin{frame}
    \frametitle{Deliverables}

    \begin{enumerate}

        \item methods to analyse behavioural and electrophysiological
        recordings from NaLoDuCo experiments that are \textbf{online} and
        \textbf{adaptive to non-stationarity} in measurements.

        \begin{scriptsize}
        For behavioural data, we will investigate methods to

        \begin{itemize}

			\begin{scriptsize}
            \item track multiple body parts of animals (deep neural networks)

            \item infer kinematics of foraging mice (linear dynamical systems)

            \item segment behaviour into discrete states

            \item characterise short- and long-term periodicities in behaviour

            \item infer the rules that govern mice behaviour from behavioural
            observations only] (i.e., policy inference).
			\end{scriptsize}

        \end{itemize}

        For neural data, we will investigate methods to:
        \begin{itemize}

			\begin{scriptsize}
            \item estimate low-dimensional continual representations of neural
            activity (i.e., latents inference)

            \item segment neural activity into discrete states

            \item characterise short- and long-term periodicities in neural
                activity

            \item decode environment variables from neural activity
			\end{scriptsize}

        \end{itemize}
        \end{scriptsize}

        \item integration of these methods into DANDI, as in Dendro, so that users
        can run them on NaLoDuCo datasets stored in DANDI.

    \end{enumerate}

\end{frame}

\begin{frame}
    \frametitle{Previous work}

    \begin{enumerate}

		\item At the Gatsby we have invented several methods for the
		characterisation of neural time
		series~\citep[e.g.,][]{yuEtAl09,dunckerAndSahani18,ruttenEtAl20,yuEtAl24,buesingEtAl12a,buesingEtAl12b,mackeEtAl15,soulatEtAl21,walkerEtAl23,turnerAndSahani14,osheaEtAl22,pachitariuEtAl13a,pachitariuEtAl13b}

		\item \href{https://github.com/magland/dendro}{Dendro} allows to
		perform advanced data analysis on DANDI. It allows to reuse previous
		analysis.

    \end{enumerate}

\end{frame}

