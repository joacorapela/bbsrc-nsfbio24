\subsubsubsection{Real-Time Machine Learning in Neuroscience}
\label{sec:rtmlNeuro}

Real-time machine learning (RTML) is widely used in domains such as finance, logistics, and environmental monitoring. For instance, in climate science, RTML enables real-time wildfire and flood detection from satellite imagery, as well as the forecasting of extreme weather events using live radar and sensor data. In food delivery systems, RTML is used to estimate delivery times based on traffic, kitchen load, and historical data, and to dynamically optimise dispatching routes.

Surprisingly, RTML remains underutilised in neuroscience. This represents a missed opportunity—especially in the context of NaLoDuCo experimentation—where adaptive, low-latency computation could greatly enhance experimental control and scientific discovery.

In this project, we will develop and deploy RTML methods for a range of applications in NaLoDuCo experiments. While experimental work is not part of this grant, it will be supported by core funding at the SWC and AIND. We will contribute by developing RTML methods, and in turn benefit from access to state-of-the-art experimental platforms.

\paragraph{Real-Time Experimental Design Verification.}
In traditional workflows, data analysis often takes place over days or weeks after acquisition. As a result, design flaws or hardware issues are only detected post hoc—sometimes requiring repetition of entire experiments. This is especially problematic in NaLoDuCo settings, where experiments may last for weeks or months. RTML can address this by enabling online monitoring of experimental progress and data quality, allowing early identification of problems and adaptive, in-situ protocol adjustment.

\paragraph{Intelligent Neuromodulation.}
Neuromodulation—delivered optically, chemically, or electrically—is typically triggered by fixed schedules or simple threshold rules. RTML allows for more flexible and context-aware interventions. For example, a scientist may hypothesise that a peak in a latent variable (inferred in real time from prefrontal cortex activity) predicts the onset of a foraging decision. Using an online latent variable model, she forecasts the peak and triggers optogenetic inactivation just before it occurs. If the intervention disrupts foraging onset, the hypothesis gains causal support.

Such intelligent neuromodulation is critical for NaLoDuCo experiments, where internal states fluctuate significantly. Fixed stimulation regimes are often ineffective in such dynamic contexts. RTML affords researchers the ability to close-the-loop between inferred latent states and behaviour, offering potential insights into the causal dynamics of internal states that might fluctuate on the order of days, weeks or even months. 

\paragraph{Intelligent Data Storage.}
As NaLoDuCo experiments grow in richness and duration, storing all raw data becomes infeasible. RTML can help decide, in real time, which data to retain and which to discard. For instance, consider a setup with ten high-resolution cameras monitoring a mouse in a large arena. Instead of storing all video streams, a real-time tracking model could estimate the animal's location and determine the bounding boxes with which to dynamically crop the camera's sensor. When tracking confidence is high, only streams from relevant cameras are saved. When uncertainty is high, more data can be preserved for later inspection. This reduces storage load while retaining critical information.

\subsubsubsection{Bonsai and Bonsai.ML}
\label{sec:bonsai}

Bonsai is a widely adopted open-source software ecosystem for experimental control in neuroscience~\citep{lopesEtAl15}. With support from the \href{https://gow.bbsrc.ukri.org/grants/AwardDetails.aspx?FundingReference=BB\%2FW019132\%2F1}{BBSRC}, we are developing infrastructure for intelligent, closed-loop experimentation through the \href{https://bonsai-rx.org/machinelearning/}{Bonsai.ML} package.

We have already integrated several real-time ML models into Bonsai.ML, including linear regression, linear dynamical systems, hidden Markov models, and Bayesian point-process decoders. In collaboration with researchers at SWC and UCL, we are applying these models to perform real-time inference of visual receptive fields, foraging kinematics, behavioural state classification, and spatial decoding of hippocampal spiking activity.\footnote{\url{https://bonsai-rx.org/machinelearning/examples/README.html}}

However, current Bonsai.ML models assume stationarity—a strong and often invalid assumption in NaLoDuCo contexts (see Section~\ref{sec:offlineAnalysisMethods}). We will extend these methods to operate under non-stationary conditions using techniques described in Section~\ref{sec:non-stationarity}.

Bonsai is already in use at both the SWC and AIND for experimental control. Working closely with researchers at these institutes, we will apply our new RTML methods to process non-stationary data and address leading scientific questions in NaLoDuCo experiments.

All new RTML tools will be released as open-source extensions to the \href{https://bonsai-rx.org/machinelearning/}{Bonsai.ML} package.

\paragraph{Reactive Machine Learning.}

A critical challenge in neuroscience, especially for NaLoDuCo experiments, is the ability to adapt experiments online. Conventional batch methods are often too slow for online experiments, since the parameters of a model can only update after collecting and processing a large enough dataset. Real-time machine learning methods can overcome this with continual, adaptive learning procedures, yet, they can still struggle to handle reactive data streams, which are inherently parallel, asynchronous, and event-driven.

In contrast to procedural ML models that simply run inference on each new data point, reactive ML models can be formulated as an interconnected graph of streaming modules~\cite{bertDeVries}. These can subscribe to independant, upstream data sources and can handle data interruptions or dropout without disruption. Crucially, reactive models can generate ongoing inference that can be used to inform closed-loop experimental control. For instance, a Kalman filter model designed in a reactive way could subscribe to multiple sensors independantly (e.g. camera and electrophysiological recordings) and infer an animal’s latent behavioral state despite temporarily dropping frames from a camera.

Similar to how software tools for deep-learning greatly accelerated research and business opportunities, libraries for reactive ML paradigms have the potential to greatly impact neuroscience research and discovery. Tools like RxInfer~\citep{RxInfer} and Infer.NET~\citep{Infer.NET} employ message-passing algorithms to support reactive model composition and specification. Inference models can update their beliefs on-the-fly as new data arrives and can adapt seamlessly when data streams are delayed or experience unexpected interruption. Currently, none of the existing reactive ML packages are particularly well-poised to be adopted into neuroscience experiments, since they would need to be integrated into existing, complicated hardware setups which already demands significant time and resources from expert engineers. By integrating a reactive ML toolbox into software specifically designed for prototyping hardware and experimental control, such as Bonsai, it is possible to build a robust, extensible reactive ML platform that neuroscientists can readily adopt for ingesting data streams from multiple cameras, neural recording devices, and other instrumentation. 

Deploying reactive ML for neuroscience experiments could enable intelligent, agentic control of experiments and have a profound impact on the communicty. Reactive ML systems can make decisions to control the system's output based on real-time inferences about neural states, behavioral events, and contextual cues. Instead of relying on systems with hard-coded, scripted values for controlling environmental cues, systems could be designed to learn and adapt to animal’s evolving behavior over days or weeks, using real-time inference to control stimulation, or select which data streams to subscribe to dynamically.

\subsubsubsection{Deliverables}

\begin{enumerate}
    \item New real-time machine learning methods for non-stationary experimental control, integrated into the open-source \href{https://bonsai-rx.org/machinelearning/}{Bonsai.ML} package.
    \item Peer-reviewed publications co-authored with SWC and AIND researchers, showcasing scientific advances made possible by the new RTML capabilities.
\end{enumerate}

