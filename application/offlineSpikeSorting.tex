
Extracting single-unit spiking activity from raw extracellular recordings
collected continuously over weeks to months presents substantial technical
challenges for which no fully adequate solutions currently exist.

First, neural waveforms undergo gradual shifts over time due to movement
between the electrode and brain tissue—a phenomenon commonly referred to as
electrode drift. In addition, neuronal firing pattens, spike waveform
shapes, and recording noise evolve over time due to circadian rhythms,
changes in animal state, and electrode degradation, resulting in
highly non-stationary signals.

Second, traditional spike sorting pipelines typically rely on manual
curation, such as deciding whether to split or merge spike waveform clusters.
While feasible for short recordings, manual curation becomes impractical for
long-duration datasets, making the development and deployment of robust
automated quality control and curation tools essential.

Third, the scale of the data poses serious computational constraints. NaLoDuCo recordings can reach hundreds of terabytes per experiment, which cannot be efficiently processed on a single GPU or workstation using conventional approaches.

To address these challenges, we propose a multi-pronged strategy:

\begin{itemize}

	\item \textbf{Experimentally}, we will minimize electrode-tissue movement
by rigidly affixing Neuropixels probes to the skull, following techniques
described in \citet[][Supplementary Discussion]{schoonoverEtAl21}.

	\item \textbf{Statistically}, we will extend existing spike sorting
algorithms and, where needed, develop new methods specifically designed to
handle the non-stationarity inherent in long-term neural recordings.

	\item \textbf{Computationally}, we will leverage distributed and
GPU-accelerated computing to scale spike sorting to datasets spanning weeks or
months. Sorting these recordings on a single machine would be prohibitively
slow; distributed infrastructure will enable timely and scalable processing.

\end{itemize}

\begin{comment}
% \subsbusubsection{Challenges}


Extracting single-unit spiking activity from raw data collected over weeks and
months of continuous extracellular recordings presents significant challenges
for which there is currently no adequate solution.
%
First, neural waveforms shift gradually over days due to electrode-tissue
movement. This problems is commonly referred as electrode drift.
%
Also, neuronal firing patterns, waveforms, and recording noise evolve due to
circadian rhythms, animal state changes, and electrode aging, generating highly
non-stationary measurements.
%
Second, conventional approaches to spiking require manual curation to, for
example, decide whether to split or merge spike waveform clusters. For
long-duration recordings manual curation is impractical, and automated curation
methods become mandatory.
%
And third, hundreds of terabytes of raw data are difficult to handle on single
GPUs or single computers.

% \subsbusubsection{Potential solutions}

To address these challenges we will attempt a multi-faceted solution.
%
Experimentally, we will cement recording probes to the skull of mice, as
described in \citet[][supplementary discussion]{schoonoverEtAl21}, to minimise
electrode-tissue movement.
%
Statistically, we will extend existing methods and, if necessary, develop
new ones to tackle the problem of non-stationarities in neuronal measurements.
%
Computationally, spike sorting month-long neural recordings on a single computer is a
prohibitedly long process. To accelerate it we will use distributed and GPU
accelerated computing.

\end{comment}

\subsubsubsection{Methods for Addressing Non-Stationarities in NaLoDuCo Recordings}

Most existing spike sorting methods for high-density probes, such as
Neuropixels, are optimized for short-duration recordings, typically spanning
only a few hours~\citep{pachitariuEtAl24,chungEtAl17,ygerEtAl18}. These methods
primarily address short-timescale electrode drift, often assuming a relatively
stable recording environment~\citep{steinmetzEtAl21}.

Only a few methods have been developed to handle the unique challenges posed by
long-duration spike sorting. Two notable approaches are:

\paragraph{UnitMatch~\citep{vanBeestEtAl24}.}  
UnitMatch is designed to stitch together spike sorting results from temporally
separated (non-continuous) recording sessions. It operates on the outputs of
any base spike sorting algorithm and identifies consistent neuronal units
across days or weeks by comparing spike templates. While highly effective at
tracking neurons across large temporal gaps, UnitMatch was not developed to
handle fully continuous data streams or to model dynamic appearance and
disappearance of neuronal units.

\paragraph{FAST~\citep{dhawaleEtAl17}.}  
FAST (Feature-Assisted Spike Tracker) is designed to track neurons over
long-duration, continuous tetrode recordings. It addresses variability in
waveform shape, firing rate, and signal quality by clustering spikes in short
temporal windows and linking cluster centroids across time. Though capable of
real-time execution, FAST was evaluated only in offline mode. While promising,
it was designed for low-channel count recordings, and adaptations are required
to extend it to high-density probes like Neuropixels.

\paragraph{Beyond Drift: Broader Non-Stationarities.}  
Although most spike sorting methods focus on correcting for electrode drift,
NaLoDuCo recordings exhibit many other sources of non-stationarity that must be
addressed:

\begin{description}

	\item[Tissue-probe interaction:] Chronic implants can cause glial scarring
or inflammation, which change the electrode impedance and reduce
signal-to-noise ratios over time.

	\item[Electrode degradation:] Long-term use may degrade probe quality due
to corrosion, microfractures, or biological encapsulation, leading to reduced
recording fidelity.

	\item[Neural plasticity:] Learning and adaptation over time can change
neuronal firing rates and patterns. Neurons may transiently disappear or
re-emerge as circuits remodel.

	\item[Brain state fluctuations:] Transitions between arousal, sleep, quiet
wakefulness, and exploration induce global shifts in firing rate distributions
and noise levels. These changes can suppress or enhance activity in entire
neural populations over hours or days.

\end{description}

\paragraph{Our Approach.}  

To address the diversity of non-stationarities in NaLoDuCo recordings, we will
evaluate and further develop both FAST and UnitMatch:

\begin{itemize}

	\item \textbf{Extending UnitMatch.} While UnitMatch can track stable units
across non-contiguous sessions, it cannot currently handle dynamic appearance
or disappearance of units. We will extend it to address these cases, allowing
tracking through long-term circuit reconfigurations and brain state
transitions.

	\item \textbf{Adapting FAST.} Originally designed for tetrode recordings,
FAST will be adapted to operate on high-channel-count Neuropixels data,
enabling robust unit tracking in continuous recordings.

	\item \textbf{Automated Quality Control.} We will apply
BombCell~\citep{fabreEtAl23} to label units (e.g., good, multi-unit, noise) and
assess their stability and isolation automatically, minimizing the need for
manual curation.

	\item \textbf{Benchmarking Base Sorters.} For UnitMatch, we will
systematically benchmark different underlying spike sorting algorithms (e.g.,
Kilosort variants) using the SpikeInterface framework~\citep{buccinoEtAl20}.

\end{itemize}

\begin{comment}
\subsubsubsection{Methods addressing non-stationarities in NaLoDuCo recordings}

Most current spike sorting methods for high-count probes, like Neuropixels,
target shorter-duration experiments, of at most a few
hours~\citep{pachitariuEtAl24,chungEtAl17,ygerEtAl18}.
%
% Most of these methods address the electrode-drift problem for short-duration
% experiments~\citep{steinmetzEtAl21}.

Two methods have been developed to spike sort recordings from long-duration
experiments.
%
First, UnitMatch~\citep{vanBeestEtAl24}, designed to
sort spikes in long-duration, but not contiguous, recordings.
%
It targets the problem of stitching together spike-sorting results from shorter
sessions, obtained using any base spike sorting method.
%
% Using Neuropixels recordings, UnitMatch reliably tracked single neurons from
% different brain regions across weeks.
%
Second, FAST~\citep{dhawaleEtAl17}, which handles long-duration and
continual tetrode recordings.
%
It addresses challenges such as significant data volume, variability in
neuronal firing rates, and temporal changes in spike waveforms that complicate
traditional spike-sorting methods.
%
It does this by clustering spike waveforms over short time windows and tracing
clusters centroids across time. This method can operate on real time, although
\citet{dhawaleEtAl17} only evaluated it for offline processing.

Current spike-sorting methods only address non-stationarities due
to eletrode drift. However, NaDoLuCo recordings display many other types of
non-stationarities as

\begin{description}


    \item[Tissue-probe interaction over time:] Glial scarring or inflammation
        can alter electrode impedance and signal amplitude, which results in
        gradual loss or transformation of signal quality (e.g., reduced SNR).

    \item[Electrode degradation or impedance change:] Over weeks to months, the
        recording quality can degrade due to microfractures in probes,
        corrosion, glial scarring or inflamation.

    \item[Neural plasticity:] Long-term recordings capture learning,
        adaptation, and memory. Neurons may change their firing rates and
        Synaptic strengths. Network dynamics can change and neurons may even
        disappear or reappear for prolonged periods of time as circuits change.

    \item[Brain state fluctuations:] Awake, drowsy, sleep (REM/NREM), quiet
        wakefulness, exploration alter spike statistics and background noise
        levels. They can completely inhibit whole circuits and make neurons
        dissapear or reappear for extended time intervals as brain states
        fluctuate.

\end{description}

To spike sort NaLoDuCo recordings offline we will evaluate FAST and UnitMatch.
%
However, UnitMatch cannot address the above non-stationarities; for example, it
cannot handle neurons that appear and dissappear periodically. We will extend
it to do so.
%
And FAST was designed for tetrodes recordings, so we will adapt it to process
recordings from Neuropixels probes.

For automatic quality control and automatic unit labeling (e.g., good, bad,
multimunit) we will use BombCell~\citep{fabreEtAl23}.

For UniMatch, we will benchmark different base methods using
SpikeInterface~\citep{buccinoEtAl20}.

% Due to these non-stationarities units may dissappear and appear a few days
% later. These dynamics are not accounted by UnitMatch and we will develop method
% to addrss them.
% 
% However, the continuity of NaLoDuCo recordings does not require matching units
% across experimental and methods building on this continuity should perform
% better than methods not using it.

% \citep{dhawaleEtAl17} developed a method to spike sort units in week-long continuous
% tetrode recordings that could be used in real time, although the authors only
% evaluated this method offline.
% 
% However, long-duration experiments cause small electrode drifts, over extended
% periods of time, which cannot be address with previous drift-correction methods
% for shorter recordings.
\end{comment}


\subsubsubsection{Accelerating spike sorting for NaLoDuCo recordings}

The UnitMatch approach spike sorting of very long-duration recordings is
highly parallelizable, as different recordings chunks can be sorted in parallel
across multiple computers, and then stitched together.

Due to its serial dependencies FAST is less parallelisable than UNitMatch.
Still, the use of GPUs should greatly accelerate it.

\subsubsubsection{Spike sorting on the cloud}

Spike sorting should be done on the cloud, or in institutional clusters. For
this we will provide a web based interface for users to specify the spike
sorting algorithm, its parameters, initiatete the spike sorting process,
monitor its progress, view its results, curate them and perform quality
control.

The AIND is performing all data analysis, including electrophysiological
processing, to the cloud. For short duration experiments, it has already
developed the above functionality, as a cloud-based implementation of
SpikeInterface.
%
We will extend this functionality to operate on NaLoDuCo experimental datasets.
This includes developing web-based visualisation tools
spanning weeks to months. For this we will use visualisation technologies similar to those
described in Section~\ref{sec:visContinuous}.

We will develop distributed verions of sorting algorithms allow them to run at
speed with NaLoDuCo datasets.

\subsubsubsection{Deliverables}

\begin{enumerate}
    % \item Benchmarking of offline spike sorting methods for NaLoDuCo recordings
    \item Methods to overcome non-stationarities in NaLoDuCo recordings.
    \item Web-based interface to allow users to spike sort NaLoDuCo recordings
        on the cloud, or in institutional computer clusters, view results of
        spike sorting, curate them and perform quality control.
    \item Server side software to run spike sorting tasks in parallel using
        distributed computing and GPUs.
    % \item New spike sorting methods

\end{enumerate}

