\subsection{Context}

Conventional system neuroscience experiments are generally short and heavily
constrain the behaviour of their subjects in order to simplify behavioural and
neural analysis.
%
However, crucial aspects of behaviour and brain function may not be expressed in
these short and constrained experiments.

At the Sainsbury Wellcome Centre (SWC) for Neural Circuits and Behaviour we are
performing Naturalistic, Long-Duration, and Continual (NaLoDuCo) foraging
experiments in mice, lasting for weeks to months, while we are recording
high-resolution behavioural and neural activity. At the Gatsby Computational
Neuroscience Unit (GCNU) we are investigating methods to analyze the new type
of data generated by these experiments.

This type of experimentation is essential for the advancement of Neuroscience.
We want to help research centres around the world to use it. However, the size
and complexity of the datasets generated by this new type of experimentation
creates important challenges in data collection and management, open data
dissemination, spike sorting, data visualization, and data analysis.  Here we
will address these challenges and share our results openly.

Hence, the main goal of the proposed project is to develop platform
technologies to help researchers build NaLoDuCo experiments and extract meaning
from the generated data.

enable a new type of
naturalistic, long-duration and continual experimentation
(Figure~\ref{fig:enablingNaLoDuCo}). To this end we will openly share data
generated in our NaLoDuCo foraging experiments, so that scientists around the
world can analyze this data and contribute to this new type of experimentation,
without having to collect data. In addition, we will
disseminate expertise and software to:

\begin{enumerate}

    \item build small animal foraging NaLoDuCo experiments,

    \item distribute large volumes of data generated by these experiments,

    \item visualize large datasets from NaLoDuCo experiments,

    \item perform spike sorting on very long-duration electrophysiological
        recordings,

    \item analyze uniquely complex data generated by these experiments with
        advanced machine learning methods,

    \item perform a new type of inference-driven experimentation.

\end{enumerate}

\subsection{Our team}

Since the begining of foraging experiments in 2021, NeuroGEARS Ltd, London, UK,
has been a key business partner building the hardware and software
infrastructure for NaLoDuCo experimentation.

The Distributed Archive for Neurophysiology Data Integration (DANDI) is
becoming a standard repository for neurophysiology data.
%
CatalystNeuro, Wyoming, US, has played a pivotal role in supporting the
development and operations of the DANDI archive.
%
Dr.~Ben Ditcher, founder of CatalystNeuro, will oversee the dissemination of
the foraging experimental data on DANDI.
%
Dr.~Jeremy Magland, senior data scientist at the Flatiron Institute, New York,
US, has ample experience on data visualization, spike sorting and cloud
computing. He will supervise a data scientist at CatalystNeuro on the
development of cloud-based data visualization and spike-sorting functionality
for NaLoDuCo experimental data stored in DANDI.

\subsection{Focus areas and their challenges}

Below we delineate the focus areas that we propose to address and outline their
challenges.
%
These challenges are mainly related to managing continuously recorded very
large datasets, on the order of hundreads of terabytes, from experiments
lasting for weeks to months.

In Neurosciences naturalistic and/or long duration and/nor continuous
experiments have been performed (e.g., ). However, to the best of our
knowledge, none of these experiments has simultanously combined these three
characteristics. Therefore, the challenges described below are unique to
NaLoDuCo experimentation.

\subsubsection{Data acquisition and management}

We have already performed foraging experiments in mice lasting three weeks,
collecting continuously behavioral and experimental data.
%
We will share openly the specifications of the hardware used in these
experiments (e.g., instructions for building large foraging arenas, video
cameras specifications, electrophysiological recording hardware), as well as
the sotware we used for experimental control, data quality control, data access
and management.

The data acquisition and management software used in our project is already
publically available in GitHub
(https://github.com/SainsburyWellcomeCentre/aeon\_mecha).
%
This software is already being used by scientists at the Allen Institue for
Neural Dynamics and at Northwester University.
%
We will substantially improve its documentation to allow external users to
build their own NaLoDuCo experiments.

Challenges related to data acquisition and management include data indexing to
allow fast access to very large amount of saved data, online quality control
and alert systems to guarantee that anomalities in data collection are detected
and corrected with minimal delay, and syncrhonization between multiple data
streams.

\subsubsection{Data dissemination}

Data sets of the scale of hundreads of terabytes cannot be practically
downloaded from data repositories. This is specially true for contiguous
experiments where unique insights will be extracted by characterizing full
datasets, and not only parts of them.
%
Therefore, we will store data in DANDI along with easy to use software for
data access, visualization and analysis.
%
That is, the large dataset sizes of NaLoDuCo experiments make it impractical to
distribute data to users and require to bring users to data.
%
Fortunately, cloud technology is now mature and allows this.

Importantly, if we distributed these very large datasets to users, only those
in large research centers would be able to process them. But, by deploying data
and computing in the cloud, any person with Internet access around the world
will be able to benefit from them.
%
Storing large datasets in DANDI is free and it is possible to obtain cloud
credits from Amazon to offer free compute to academic institution.

\subsubsection{Data visualisation}

Visualisations are essential for scientific discovery.
%
For the proposed project visualisation present two major challenges. First, they need
to display very large datasets at different temporal scales, from milliseconds
to weeks and months. Second, at data and software will be deployed in the
cloud. visualisation need to be web based.

Standard visualization tools cannot display terabyte sized datasets.
%
We will build custom web-based visualization tools to do this.
%
We have substantial experience building web-based visualization tools for
neurophysiological data (e.g.,).

\subsubsection{Spike sorting}

When electrodes are placed in the brain, they typically record spikes from
multiple nearby neurons. Spike sorting attributes spikes to individual neurons.

Spike sorting is specially challenging for NaLoDuCo experiments.
%
First, because these experiments require to track individual neurons of freely
moving mice for weeks to months.
%
Second, because spike sorting needs to be done online, to allow experiments
driven by real-time machine learning inference, as described below.

Prof.~Sahani pioneered the use of Bayesian inference methods for spike
sorting~\citep{sahaniPhDThesis}.
%
Dr.~Jeremy Magland has significantly advanced the field of spike sorting,
particularly through his development of MountainSort and his contributions to
SpikeInterface.

\subsubsection{Data analysis}

\subsubsection{Experiments driven by real-time machine learning inference}

\subsection{Challenge that our project addresses}

The massive datasets generated by NaLoDuCo
experiments open new challenges for distribution and for knowledge extraction
(e.g., data analysis). For example, a one-month-long foraging experiments
generates xx~TB of behavioral data and yy-TB of electrophysiological data.
%
\textbf{The first set of challenges that our project will address is the
distribution of very large neuroscience datasets}.

Most conventional machine learning methods do not scale to the very large size
of the datasets generated by our experiments.  Another challenge in NaLoDuCo
experiments is that the statistics of the data they generate most often change
with time. Yet most current machine learning methods assume that these
statistics remain constant in time.
%
\textbf{The second challenge is to find suitable machine learning methods to
extract meaning from NaLoDuCo experiments, or develop new methods if we cannot
find suitable ones}.

\subsubsection{Experimental control, data aquisition and data management}

We will openly share descriptions of the hardware, software for experimental
control, software for data acquistion, and software management that we are
using at the SWC to perform NaLoDuCo foraging experimetnation.


%
With these hardware and software we have recorded continuous behavior fox xx
weeks, and continuous simultaneous behavior and neural activity for yy weeks.

A central aim of our experimental setup is reproducibility; we aim at building
experiments that can easily by reproduced by others.
%
We want to use free open-source software.
%
And need experimental control software that is interoperable, i.e., that can
interact with a diverse set of experimental devices.

Although some of this software and hardware will be specific to foraging, a
large part of it will be suitable for other type of experiments.
%
Indeed, our software for experimental
control\footnote{\url{https://github.com/SainsburyWellcomeCentre/aeon_experiments}},
data
acquistion\footnote{\url{https://github.com/SainsburyWellcomeCentre/aeon_experiments}}
and
management\footnote{\url{https://github.com/SainsburyWellcomeCentre/aeon_mecha/}}
is open source and it is being used at the Allen Institute for Neural Dynamics
at a Nortwestern University.

\subsubsection{Data dissemination}

\subsubsection{Data analysis}

%
We will openly share data generated by this new type of experimentation
and disseminate carefully validated statistical methods to extract insights from
these data.

\subsection{Challenge that our project addresses}

The massive datasets generated by NaLoDuCo
experiments open new challenges for distribution and for knowledge extraction
(e.g., data analysis). For example, a one-month-long foraging experiments
generates xx~TB of behavioral data and yy-TB of electrophysiological data.
%
\textbf{The first set of challenges that our project will address is the
distribution of very large neuroscience datasets}.

Most conventional machine learning methods do not scale to the very large size
of the datasets generated by our experiments.  Another challenge in NaLoDuCo
experiments is that the statistics of the data they generate most often change
with time. Yet most current machine learning methods assume that these
statistics remain constant in time.
%
\textbf{The second challenge is to find suitable machine learning methods to
extract meaning from NaLoDuCo experiments, or develop new methods if we cannot
find suitable ones}.

\subsection{Aims and objectives}

The specific aims of the project are:

\begin{enumerate}

    \item openly distribute high-resolution behavioral and electrophysiological
        recordings from mice foraging in naturalistic environments continuously
        for weeks to months.

    \item thoroughly evaluate existing machine learning methods targeting
        key problems for NaLoDuCo experimentation, focusing on online-adaptive
        methods.
        %
        Build new methods if suitable ones cannot be found.

    \item further develop inference-driven experimentation, a new type of
        experimentation controled by statistical inferences on behavioral and
        neural data (Section~\ref{sec:inferenceDrivenExp}).

    \item apply the above methods, in collaboration with scientists at the SWC,
        to investigate decision making in foraging mice.

    \item create a live benchmark for methods characterizing NaLoDuCo
        experimental data.

\end{enumerate}

\subsection{Potential applications and benefits}

This project will enable a new type of naturalistic experimentation allowing to
observe complex patterns of behavior and neural activty as they naturally
evolve over extended periods of time.

\subsection{Relevance to the BBSRC long-term research and innovation
priorities}

The proposed project is relevant to several objectives of the BBSRC
Strategic Delivery plan 2022-2024:

\begin{itemize}

    \item this award will continue the support to the recently established
        Statistical Signal Processing for Neuroscience group lead by
        Dr.~Joaquin Rapela (co-pi in this proposal) (section 1.1 People and
        talent).

    \item the main outputs of the proposed project are open data and open
        methods (section \emph{1.2 Research and Innovation Culture}).

    \item the proposed work will be essential towards understanding foraging
        behavior and its neural basis (section \emph{3.2: Understanding the
        rules of life}).

    \item NeuroGEARS Ltd is an essential partner in this proposal (section 4.1:
        Enabling innovation and working with business).

    \item we aim at enabling a new type of experimentation and
        creating new machine learning methods to process the data generated by
        the new experiments (section 5.1: Transformative technologies).

\end{itemize}
