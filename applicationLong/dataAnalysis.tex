% Context

Having collected a new type of neuroscience dataset, created cloud
infrastructure for open sharing, and developed open-source software for
visualization, we will now provide functionality to automatically extract
meaning from it.

\subsubsection*{Challenges and opportunities}

The automatic analysis of NaLoDuCo experimental data opens unique challenges an
opportunities. A few challenges are:

\begin{description}

    \item[need for online learning] The vast majority of machine learning
    algorithms process all data points at the same time; i.e., they are batch
    algorithms. This processing mode in inadequate for NaLoDuCo datasets,
    because they cannot be loaded into memory, due to their large data sizes.
    These datasets require \textbf{online learning}~\citep{shalevShwartz12}, a
    form of learning where models update their parameters one sample at a time,
    as soon as they are received.

    \item[addressing non-stationarity] Non-stationarity occurs when the statistical
    properties of data change over time, leading to a mismatch between
    previously learned models and new data distributions. In the context of
    online machine learning, where models are continuously updated with
    incoming data, non-stationarity poses a significant challenge because it
    can degrade model performance if not properly addressed.

    For example, in neuroscience experiments recording neural activity over
    weeks to months, the relationship between neural signals and behavioral
    outcomes may shift due to learning, fatigue, circadian rhythms, or
    environmental changes. A model trained on early data may become outdated as
    the neural response patterns and behaviour evolve.

    The field of adaptive signal processing\citep{haykin02} studies algorithms for
    non-stationary data.

\end{description}

\paragraph{Initial algorithms to optimize}

\paragraph{Implementation notes}
