% \subsbusubsection{Challenges}

Spike sorting of continuous, month-long Neuropixel recordings is challenging
for at least four reasons.
%
First, neural waveforms shift gradually over days due to electrode-tissue
movement. This problems is commonly referred as electrode drift.
%
Second, neuronal firing patterns, waveforms, and recording noise evolve due to
circadian rhythms, animal state changes, and electrode aging, generating highly
non-stationary measurements.
%
Third, conventional approaches to spiking require manual curation to, for
example, decide whether to split or merge spike waveform clusters. For
long-duration recordings manual curation is impractical, and automated
curation methods become mandatory.
%
And fourth, hundreds of terabytes of raw data are difficult to handle on single
GPUs or single computers.

% \subsbusubsection{Potential solutions}

To address these challenges we will attempt a multi-faceted solution.
%
Experimentally, we will cement recording probes to the skull of mice, as
described in \citep[][supplementary discussion]{schnoonoverEtAl21}, to minimise
electrode-tissue movement.

Methodologically, we will use existing methods, and try to improve them, to
tackle the problem of electrode-tissue movement and that of non-stationarity in
neuronal measurements.

Spike sorting month-long neural recordings on a single computer is a
prohibitedly long process. To accelerate it we will use distributed and GPU
accelerated computing.

\subsubsubsection{Methods addressing electrode drift and non stationarities in NaLoDuCo recordings}

Most current spike sorting methods for high-count probes, like Neuropixels
probes, target shorter-duration experiments, of at most a few
hours~\citep[e.g.,][]{pachitariuEtAl24}.
%
Most of these methods address the electrode-drift
problem for short-duration experiments~\citep{steinmetzEtAl21}.

Few methods have been developed to spike-sort recordings from long-duration
experiments.
%
One such method is UnitMatch~\citep{vanBeestEtAl24}, which is designed to to
sort spikes in long-duration, but not contiguous, recordings.
%
It targets the problem of stitching together spike-sorting results from shorter
sessions, obtained using any base spike sorting method.
%
Using Neuropixels recordings, UnitMatch reliably tracked single neurons from
different brain regions across weeks.

Another method is FAST~\citep{dhawaleEtAl17A}, which is designed to handle
long-duration and continual recordings. It does this by clustering spike
waveforms over short time windows and tracing clusters centroids across time.
This method can operate on real time. We will adapt this method for oline spike
sorting of NaLoDuCo recordings (Section~\ref{sec:onlineSpikeSorting}).

To spike sort NaLoDuCo recordings offline we will follow the approach of
UnitMatch. We will first spike sort recording chunks of size equal to the GPU
memory size using a base (short-duration) spike sorting method. These chunks
will be processed in parallel across different nodes in a distributed computing
environment.
%
We will then stitch the spike sorted chunks using UnitMatch.
%
For automatic quality control and automatic unit labeling (e.g., good, bad,
multimunit) we will use BombCell.

To implement the previous pipeline we will benchmark different spike sorting
base methods on short sessions from (a) foraging NaLoDuCo recordings acquired
at the SWC and (b) odour learning NaLoDuCo recordings acquired at the AIND. For
this we will use SpikeInterface~\citep{buccinoEtAl20}.

Most short-duration spike-sorting methods only address non-stationarities due to
eletrode drift. However, in NaDoLuCo recordings display many other types of
non-stationarities, like 
Using the UnitMatch approach spike sorting of very long-duration recordings is
highly parallelizable, as different recordings chunks can be sorted in parallel
across multiple computers, and then stitched together.
However, the continuity of NaLoDuCo recordings does not require matching units
across experimental and methods building on this continuity should perform
better than methods not using it.

\citep{dhawaleEtAl17} developed a method to spike sort units in week-long continuous
tetrode recordings that could be used in real time, although the authors only
evaluated it offline.

However, long-duration experiments cause small electrode drifts, over extended
periods of time, which cannot be address with previous drift-correction methods
for shorter recordings.

\subsubsubsection{Accelerating spike sorting in NaLoDuCo recordings}

\subsubsubsection{Deliverables}

\begin{enumerate}
    \item Benchmarking of offline spike sorting methods for NaLoDuCo recordings
    \item Method to overcome other non-stationarities in NaLoDuCo recordings, different from electrode drift (see )
    \item Web-based interface to allow users to spike sort NaLoDuCo recordings on the cloud, or in institutional computer clusters
    \item Server side software to run spike sorting tasks in parallel using distributed computing and GPUs
    \item New spike sorting methods
    \item Web-based interface to allow users to view results of spike sorting, curate them and perform quality control

\end{enumerate}

