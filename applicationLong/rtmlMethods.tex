\subsubsubsection{Real-Time Machine Learning in Neuroscience}

Real-time machine learning (RTML) is now widely used across many disciplines.

For example, in climate and environmental monitoring, RTML supports real-time detection of floods and wildfires using satellite and sensor data, and predicts extreme weather events from streaming radar and temperature data.

In food delivery, RTML predicts delivery times using live traffic, restaurant queues, and historical patterns, and dynamically plans optimal routes for drivers and shoppers.

Despite its potential, RTML remains underutilized in neuroscience. This is surprising given its promise for next-generation experiments involving navigation, learning, and dynamic control (NaLoDuCo).

\paragraph{Real-time experimental design verification.}
Neuroscience experiments typically rely on offline analysis. These analyses often uncover flaws in data collection only after experiments are complete, requiring multiple iterations.

This is impractical for long-duration experiments that span weeks or months. A better approach is to perform real-time analysis and adapt experimental protocols dynamically when issues are detected.

\paragraph{Intelligent neuromodulation.}
Brain activity can be modulated optically, chemically, or electrically. Traditionally, such interventions are scheduled at fixed times or based on simple behavioral cues.

A more sophisticated method uses inferences from machine learning to guide modulation. For example, a scientist may hypothesize that a peak in a latent neural variable (derived from prefrontal cortex activity) signals a decision to forage. By estimating this latent variable online and forecasting its peak, the scientist can optogenetically inactivate the relevant neural population just before the predicted peak. If this prevents the mouse from initiating foraging, the hypothesis is supported.

\paragraph{Intelligent data storage.}
As NaLoDuCo experiments increase in duration and data richness, storing all raw data becomes unfeasible. Intelligent real-time data pruning becomes essential.

For instance, in a large arena monitored by ten high-resolution cameras, it is more efficient to store footage only from the cameras actively observing the mouse. Real-time probabilistic tracking can guide this, saving all footage when tracking confidence is low, and selectively saving when confidence is high.

\subsubsubsection{Bonsai and Bonsai.ML}

Bonsai is a widely used software ecosystem for experimental control in neuroscience~\citep{lopesEtAl15}.

With support from the \href{https://gow.bbsrc.ukri.org/grants/AwardDetails.aspx?FundingReference=BB\%2FW019132\%2F1}{BBSRC}, we are building infrastructure for intelligent experimentation through the \href{https://bonsai-rx.org/machinelearning/}{Bonsai.ML} package.

Bonsai.ML currently includes online machine learning models such as linear regression, linear dynamical systems, hidden Markov models, and Bayesian point-process decoders. In collaboration with researchers at SWC and UCL, we apply these models to neuroscience challenges including receptive field estimation, foraging behavior analysis, behavioral state inference, and hippocampal position decoding.

However, Bonsai.ML models assume data stationarity, which is incompatible with NaLoDuCo experiments (see Section~\ref{sec:offlineAnalysisMethods}). We are adapting these models for non-stationary settings using techniques described in Section~\ref{sec:non-stationarity}.

These new machine learning modules for real-time control in non-stationary environments will be released as part of the open-source \href{https://bonsai-rx.org/machinelearning/}{Bonsai.ML} package.

At SWC and AIND, Bonsai is a core part of our experimental infrastructure. In partnership with scientists at both institutions, we will deploy these RTML methods in state-of-the-art NaLoDuCo neuroscience experiments.

\subsubsubsection{Deliverables}

\begin{enumerate}
    \item A repository of real-time machine learning methods for experimental control in non-stationary environments.
    \item Collaborative publications with SWC and AIND researchers reporting scientific findings from NaLoDuCo experiments using non-stationary Bonsai.ML methods.
\end{enumerate}

