\subsubsubsection{Real-Time Machine Learning in Neuroscience}

% ## RTML is used in many disciplines

Real time machine learning (RTML) is currently used across many disciplines.
%
For example, in climate and environmental monitoring it is used for real-time
flood or wildfire detection from satellite or sensor data, and for predicting
extreme weather events using streaming radar and temperature data.
%
Another example is food delivery where RTML is used for predicting delivery
time based on live traffic, restaurant queues, and historical patterns and
dynamic route planning for drivers and shoppers.

% ## Despite its potential in neuroscience, it has not find many applications

However, there are very few applications of RTML in neurosciences. This is
surprising, since the potential of RTML for neuroscience is enormous, specially
for NaLoDuCo experimentation.

\paragraph{Real-time experimental design verification.}
Offline analysis is the standard in neuroscience. These analyses
often reveal deficiencies in the data collection process and scientists carry
out multiple iterations of an experiment until they converge to the desired
one.
%
This modus operandi is not adequate for long-duration experimentation, as it
would not be practical to perform several iterations of months long
experiments. One solution would be to perform analysis online, and update the
experiment design if the online analysis reveals deficiencies.

\paragraph{Intelligent neuromodulation.}
%
Brain activity can be modulated optically, chemically and electrically.
%
Most commonly this modulations is done at fixed experimental times, or based on
simple behavioural or neural observations.
%
A better approach is to guide neural manipulations based on inferences from
advanced machine learning methods.
%
For example, a scientists may hypothesize that a peak in a neural latent
variable, inferred from a prefrontal cortex population, signals the moment when
mice decide to begin a foraging bout.  To test this, she estimates latent
variables online from prefrontal cortex activity, and use them to forecast when
this peak will occur. She then optogenetically inactivates the neural
population before the forecasted time.  Because inactivation prevented the
mouse from initiating a foraging bout, her hypothesis is supported.

\paragraph{Intelligent data storage.}
%
As the duration of NaLoDuCo experiments become longer, and the richness of the
behavioural and neural recordings become larger, it will be unfeasible to
store all raw data. We will be forced to intelligently decide, in real time,
subsets of data to discard.

For instance, if we are recording videos from a mouse foraging in a large arena
with ten high-resolution cameras, it would save considerable storage if at any
time we only save videos from cameras capturing the mouse.  This
could be done by tracking the position of the mouse in real time with
probabilistic machine learning methods. Then, when the confidence of the
tracking is high, we would only save videos of cameras capturing the mouse at
the tracked position, but when the confidence is low, we would save all videos.

\subsubsubsection{Bonsai and Bonsai.ML}

Bonsai is a software ecosystem for neuroscience experimental control used by
thousands of scientists around the world~\citep{lopesEtAl15}.
%
Funded by the
\href{https://gow.bbsrc.ukri.org/grants/AwardDetails.aspx?FundingReference=BB\%2FW019132\%2F1}{BBSRC}
we are building software infrastructure to enable intelligent experimentation
in the \href{https://bonsai-rx.org/machinelearning/}{Bonsai.ML}
package.
%
We have integrated into
\href{https://bonsai-rx.org/machinelearning/}{Bonsai.ML} several online machine
learning models (e.g., linear regression, linear dynamical systems, hidden
Markov models, Bayesian point-process decoders) and, in collaboration with scientists at the
SWC and UCL, we are applying these models to neuroscience problems (e.g.,
estimation of visual receptive fields, inference of foraging mice kinematics,
inference of behavioral states, position decoding from hippocampal mice
activity).

\href{https://bonsai-rx.org/machinelearning/}{Bonsai.ML} methods assume
stationarity that, as discussed in Section~\ref{sec:offlineAnalysisMethods}, is
not suitable for NaLoDuCo experimentation. We will adapt these methods to
operate in non-stationary environments using the techniques outlined in
Section~\ref{sec:non-stationarity}.

The new advanced machine learning methods for intelligent experimental control in
non-stationary environments will be openly disseminated as new modules of the
\href{https://bonsai-rx.org/machinelearning/}{Bonsai.ML} package.

At the SWC and at the AIND we use Bonsai for experimental control. In
collaboration with scientists at both institutes, we will use the new RTML
methods to process non-stationary data in state-of-the-art
neuroscience NaLoDuCo experiments.

\subsubsubsection{Deliverables}

\begin{enumerate}

    \item repository of real time ML methods for neuroscience experimental
        control adapted to work in non-stationary environments.

    \item publications with scientists at the SWC and AIND reporting findings
        in NaLoDuCo experiments using non-stationary Bonsai.ML methods.

\end{enumerate}
