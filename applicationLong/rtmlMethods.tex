\subsubsubsection{Real-Time Machine Learning in Neuroscience}
\label{sec:rtmlNeuro}

Real-time machine learning (RTML) is widely used in domains such as finance, logistics, and environmental monitoring. For instance, in climate science, RTML enables real-time wildfire and flood detection from satellite imagery, as well as the forecasting of extreme weather events using live radar and sensor data. In food delivery systems, RTML is used to estimate delivery times based on traffic, kitchen load, and historical data, and to dynamically optimise dispatching routes.

Surprisingly, RTML remains underutilised in neuroscience. This represents a missed opportunity—especially in the context of NaLoDuCo experimentation—where adaptive, low-latency computation could greatly enhance experimental control and scientific discovery.

In this project, we will develop and deploy RTML methods for a range of applications in NaLoDuCo experiments. While experimental work is not part of this grant, it will be supported by core funding at the SWC and AIND. We will contribute by developing RTML methods, and in turn benefit from access to state-of-the-art experimental platforms.

\paragraph{Real-Time Experimental Design Verification.}
In traditional workflows, data analysis often takes place days or weeks after acquisition. As a result, design flaws or hardware issues are only detected post hoc—sometimes requiring repetition of entire experiments. This is especially problematic in NaLoDuCo settings, where experiments may last for weeks or months. RTML can address this by enabling online monitoring of experimental progress and data quality, allowing early identification of problems and adaptive, in-situ protocol adjustment.

\paragraph{Intelligent Neuromodulation.}
Neuromodulation—delivered optically, chemically, or electrically—is typically triggered by fixed schedules or simple threshold rules. RTML allows for more flexible and context-aware interventions. For example, a scientist may hypothesise that a peak in a latent variable (inferred in real time from prefrontal cortex activity) predicts the onset of a foraging decision. Using an online latent variable model, she forecasts the peak and triggers optogenetic inactivation just before it occurs. If the intervention disrupts foraging onset, the hypothesis gains causal support.

Such intelligent neuromodulation is critical for NaLoDuCo experiments, where internal states fluctuate significantly. Fixed stimulation regimes are often ineffective in such dynamic contexts.

\paragraph{Intelligent Data Storage.}
As NaLoDuCo experiments grow in richness and duration, storing all raw data becomes infeasible. RTML can help decide, in real time, which data to retain and which to discard. For instance, consider a setup with ten high-resolution cameras monitoring a mouse in a large arena. Instead of storing all video streams, a real-time tracking model could estimate the animal's location. When tracking confidence is high, only streams from relevant cameras are saved. When uncertainty is high, more data can be preserved for later inspection. This reduces storage load while retaining critical information.

\subsubsubsection{Bonsai and Bonsai.ML}
\label{sec:bonsai}

Bonsai is a widely adopted open-source software ecosystem for experimental control in neuroscience~\citep{lopesEtAl15}. With support from the \href{https://gow.bbsrc.ukri.org/grants/AwardDetails.aspx?FundingReference=BB\%2FW019132\%2F1}{BBSRC}, we are extending this framework through the \href{https://bonsai-rx.org/machinelearning/}{Bonsai.ML} package to support intelligent, closed-loop experimentation powered by real-time machine learning (RTML).

We have already integrated several foundational RTML models into Bonsai.ML, including linear regression, linear dynamical systems, hidden Markov models, and Bayesian point-process decoders. In collaboration with researchers at SWC and UCL, we are applying these models to real-time inference tasks, including visual receptive field estimation, foraging kinematics decoding, behavioral state classification, and spatial decoding from hippocampal spiking activity.\footnote{\url{https://bonsai-rx.org/machinelearning/examples/README.html}}

However, these existing models assume stationarity—an assumption often violated in NaLoDuCo experiments, where data distributions evolve over long time periods (see Section~\ref{sec:offlineAnalysisMethods}). We will adapt current methods to non-stationary settings using the techniques discussed in Section~\ref{sec:non-stationarity}.

Linear models from adaptive signal processing, such as least mean squares and recursive least squares, are already suited for real-time use and require no major modification. Nonlinear regression, particularly using artificial neural networks (ANNs), can support real-time inference (especially on GPUs), but real-time learning is more challenging. Continual learning methods—such as experience replay and online regularization—introduce additional computational overhead (e.g., data buffering, importance weight tracking) that may exceed the constraints of real-time systems.

To address these constraints, we will explore several adaptation strategies:

\begin{description}
    \item[Real-time minibatch updates:] Training is performed on small minibatches as data arrive, balancing learning speed with latency.
    \item[Bounded-time updates:] Only a subset of model parameters (e.g., output layer weights) is updated to reduce computation time.
    \item[Hybrid strategy:] A dual-pipeline system in which a lightweight model performs fast inference, while a slower process (running on a separate processor or node) accumulates recent data in a rolling buffer and periodically retrains or updates the model. The updated model is deployed into the fast path only after validation, ensuring continuity in inference.
\end{description}

State-space models and their adaptive variants are particularly amenable to real-time application. Nevertheless, in non-stationary contexts, these too may benefit from hybrid strategies for continuous adaptation.

For more complex models, concept drift methods can be integrated into the hybrid architecture. For example, in a real-time spatial decoding task from non-stationary hippocampal signals, an ensemble of decoders trained on different recent time windows (e.g., one day, three days, one week) could be maintained. One model from the ensemble performs real-time inference, and its performance is continuously monitored. If its decoding accuracy deteriorates (e.g., due to drift), the system evaluates the other models in the ensemble and switches to the best-performing one—without interrupting real-time processing. This approach ensures both robustness and adaptability in changing environments.

Bonsai is already used extensively for experimental control at the SWC and AIND. Collaborating with researchers at these institutes, we will deploy our RTML enhancements to address major scientific challenges in NaLoDuCo experiments.

All newly developed RTML tools for non-stationary experimental control will be released as open-source extensions to the \href{https://bonsai-rx.org/machinelearning/}{Bonsai.ML} package.

\subsubsubsection{Deliverables}

\begin{enumerate}
    \item Real-time machine learning methods for non-stationary experimental control, integrated into the \href{https://bonsai-rx.org/machinelearning/}{Bonsai.ML} package.
    \item RTML enhancements supporting continual learning, concept drift adaptation, and hybrid fast/slow architectures.
    \item Demonstration of these methods in NaLoDuCo experiments at SWC and AIND.
    \item Peer-reviewed publications co-authored with SWC and AIND researchers, showcasing new scientific discoveries enabled by RTML.
    \item Open-source release of all code, usage examples, and documentation.
\end{enumerate}

\subsubsubsection{Bonsai and Bonsai.ML}
\label{sec:bonsai}

Bonsai is a widely adopted open-source software ecosystem for experimental
control in neuroscience~\citep{lopesEtAl15}. With support from the
\href{https://gow.bbsrc.ukri.org/grants/AwardDetails.aspx?FundingReference=BB\%2FW019132\%2F1}{BBSRC},
we are developing infrastructure for intelligent, closed-loop experimentation
through the \href{https://bonsai-rx.org/machinelearning/}{Bonsai.ML} package.

We have already integrated several real-time ML models into Bonsai.ML,
including linear regression, linear dynamical systems, hidden Markov models,
and Bayesian point-process decoders. In collaboration with researchers at SWC
and UCL, we are applying these models to real-time inference of visual
receptive fields, foraging kinematics, behavioural state classification, and
spatial decoding of hippocampal spiking
activity.\footnote{\url{https://bonsai-rx.org/machinelearning/examples/README.html}}

However, current Bonsai.ML models assume stationarity—a strong and often
invalid assumption in NaLoDuCo contexts (see
Section~\ref{sec:offlineAnalysisMethods}). We will adapt these methods to
operate under non-stationary conditions using techniques described in
Section~\ref{sec:non-stationarity}.

Linear regression algorithms from adaptive signal processing, like
least-mean square or recursive least square, are designed to operate in real
time, so they will require no adaptation for real-time operation.

For non-linear regression, artificial neural networks can perform inference in
real time, especially when deployed on GPUs. However, for learning, techniques from continual
learning, like experience replay  or online regularization, are not designed
for real-time operation. They require additional computations, like storing and
replaying data or computing weight importance, which might not meet real-time
constrains, depending on model size, available compute, or latency
requirements.
%
To try to overcome this limitation we could use:

\begin{description}

    \item[real-time minibatch updates] where parameters are updated on small
        minibatches.

    \item[bounded-time online updates] where only a subset of weights are
        updated (e.g. input or output weights).

    \item[hybrid strategy] that separates computation into two parallel
        pipelines (running on difference processors or computers): a fast path
        for real-time inference and a slow path for background model
        adaptation. The fast path uses a lightweight, pre-trained model to
        perform inference on incoming data at millisecond latencies.
        Concurrently, the slow path accumulates recent data in a rolling buffer
        and periodically updates or retrains models.  This path incorporates
        methods from continual learning and concept drift adaptation, enabling
        the system to remain accurate over long timescales. When an updated
        model is validated, it is seamlessly deployed into the fast path
        without interrupting real-time performance.

\end{description}

State-space models and their adaptive modifications are well suited to operate
on real time, but the previous hybrid strategy may be needed to adapt
model parameters to non-stationary regimes.

For other types of machine learning models, techniques from the concept drift
literature, combined with the previous hybrid strategy, could be used for
real-time inference and learning.
%
For example, suppose we need a real-time decoder of position from
non-stationary hippocampal activity. We could use an ensemble of decoders
trained on different time windows (e.g., last week, last three days,
yesterday). Among these models, one would be used to perform inference, and we
will monitor its real-time performance. A performance degradation (e.g.,
increase in decoding error) would trigger an model update, where we would pick
the best performing model of the ensemble, and switch over to this model
without interrupting real-time inference.

Bonsai is already in use at both the SWC and AIND for experimental control. Working closely with researchers at these institutes, we will apply our new RTML methods to process non-stationary data and address leading scientific questions in NaLoDuCo experiments.

All new RTML tools will be released as open-source extensions to the \href{https://bonsai-rx.org/machinelearning/}{Bonsai.ML} package.

\subsubsubsection{Deliverables}

\begin{enumerate}
    \item New real-time machine learning methods for non-stationary experimental control, integrated into the open-source \href{https://bonsai-rx.org/machinelearning/}{Bonsai.ML} package.
    \item Peer-reviewed publications co-authored with SWC and AIND researchers, showcasing scientific advances made possible by the new RTML capabilities.
\end{enumerate}

