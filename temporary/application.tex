
\documentclass[12pt]{article}

\title{Enabling naturalistic, long-duration and continual animal experimentation}

\begin{document}

\maketitle

\section{Vision}

For over four years, at the Sainsbury Wellcome Centre, we have been developing
the AEON platform, a set of hardware and software tools that support possible a
new type of experimentation, where animals are allowed to express
ethologically-relevant behaviors in naturalistic conditions, in long-duration
experiments, while detailed behavioral and neural measurements are measured for
weeks to months.

We have used this platform to characterize foraging behavior in both solitary
and groups of mice~\cite{aeonRepo}.
%
Our US partner, the Allen Institute for Neural Dynamics, is using the AEON
platform in continuous learning experiments, where mice freely explore odors
continuously for days to weeks~\cite{carlsPapers}..

This is an unprecedented type of experimentation that \ldots

Several groups around the world are developing this new type of
experimentation~\cite{}.

\subsection{Aim 1: create software to support visualisation and advanced data
analysis of NaLoDuCo experimental data on the cloud}

Using the AEON platform we and others have recorded unprecedented data. We next
want to openly disseminate these data. However, this dissemination is not
trivial, as datasets generated by this new type of experimentation are
enormous. For instance, the size of a dataset generated from a one week
recording of behavioral and neural activity from a foraging mouse in SWC
experiments exceeds 200 terabytes. It will take users several days to download
these datasets over standard Internet connections.
%
So, instead of bringing data to users, we need to bring users to data, by
storing dataset from NaLoDuCo experiments in the cloud, and providing software
running on the cloud to allow users to visualise and analyse these datasets
where they live.
%
The first aim of this grant is to create this software (Figure~\ref{fig:aims},
left).

\subsection{Aim 2: creating infrastructure for real-time neuroscience
experimentation}

As the duration of NaLoDuCo experiments became longer, and the richness of the
behavioral and neural recordings became larger, it will become unfeasible to
store all raw data. We will be forced to intelligently decide, in real time,
subsets of data to store. This is one important application of real-time
machine learning to NaLoDuCo experimentation.

For instance, if we are recording videos from a mouse foraging in a large arena
with ten high-resolution cameras, it would save considerable storage to only
save videos from cameras capturing the mouse at a given time point. This could
be done by tracking the position of the mouse in real time with probabilistic
machine learning methods. Then, when the accuracy of the tracking is high, we would only save
videos of cameras capturing the mouse at the tracked position, but when the
accuracy of the tracking is not high, we would save all videos.

save recordings from cameras not focused on the mouse
This proposal has two main aims. First,
to 
\end{document}
