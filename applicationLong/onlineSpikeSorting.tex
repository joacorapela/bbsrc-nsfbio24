Few methods currently support online spike sorting with high-channel-count probes such as Neuropixels. Nevertheless, this is an active area of research, and new techniques are rapidly emerging. In this project, we will evaluate state-of-the-art online spike sorting algorithms on NaLoDuCo foraging datasets collected at the SWC, and on olfactory learning datasets collected at the AIND.

Two key methods we will evaluate are \textbf{FAST} and \textbf{RT-Sort}.

\paragraph{FAST}~\citep{dhawaleEtAl17}, described in Section~\ref{sec:offlineSpikeSorting}, was originally developed for long-term tetrode recordings. While the original implementation was evaluated offline, FAST is computationally efficient and amenable to online adaptation. We will explore the feasibility of adapting FAST to high-density probes and implementing it for real-time processing. The method clusters spikes in short time windows and links clusters across time via a chaining algorithm, enabling robust tracking of units under waveform drift and neural plasticity.

\paragraph{RT-Sort}~\citep{vanDerMolenEtAl24} is a real-time spike detection and sorting algorithm developed specifically for high-density extracellular recordings. It achieves low-latency processing suitable for closed-loop applications. RT-Sort operates in two stages:

\begin{itemize}
    \item First, it uses a lightweight \textbf{convolutional neural network (CNN)} to continuously detect spikes in streaming multichannel voltage data.
    \item Second, it performs spike sorting based on the \textbf{spatiotemporal propagation patterns} of the detected spikes. These patterns—defined by the timing of voltage extrema across channels—are compared against a dynamic library of templates, enabling robust online unit classification. Templates are updated continually as new spikes are observed, allowing the method to adapt to changes in waveform shape over time.
\end{itemize}

RT-Sort is GPU-accelerated and optimised for real-time performance, making it particularly attractive for use in NaLoDuCo experiments involving adaptive stimulation or closed-loop behavioural control.

As online spike sorting will become essential for future adaptive experiments, we will benchmark these methods in terms of speed, accuracy, and robustness to non-stationarities, and integrate them into our experimental workflows when feasible.

\subsubsubsection{Deliverables}

\begin{enumerate}
    \item A comprehensive benchmarking of current online spike sorting algorithms—including FAST and RT-Sort—on NaLoDuCo datasets, evaluating accuracy, latency, and robustness to non-stationarities.
    
    \item Adaptation and optimisation of one or more online spike sorting pipelines (e.g., FAST) for real-time operation on high-density probes like Neuropixels.

    \item Integration of validated online spike sorting methods into existing experimental workflows at the SWC and AIND, enabling closed-loop control and real-time experimental feedback.

    \item Open-source dissemination of code and tools for real-time spike sorting, including documentation and example pipelines using NaLoDuCo data.
\end{enumerate}

