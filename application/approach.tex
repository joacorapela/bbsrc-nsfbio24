\subsubsection{Data collection \& management}

We have developed a new platform that allows housing of mice in large arenas
(\textgreater 2m diameter), while manipulating and monitoring their behaviour
at high spatiotemporal resolution \citep[Figure~\ref{fig:arena}, ][]{campagnerEtAl24}.
%
We have openly shared software for supporting data acquistion
\citep{aeonacquisition} and management \citep{aeonmecha} in this
arena.
%
Using this platform we have collected several week long datasets both with
single mouse and multiple mice.
%
These datasets capture a rich behavioural repertoire including a foraging
behaviour,  social learning task, defensive and nesting behaviours.


\begin{figure}
    \centering
    \subfloat[]{
        \includegraphics[width=4in]{figures/arena.png}
    }
    \hfill
    \subfloat[]{
        \includegraphics[width=4in]{figures/patch.png}
    }
    \caption{Foraging Arena (a) and Feeder (b).
    %
    The arena is composed of tessellated hexagonal tiles (a), each featuring a
    newly designed underground feeder (b).
    %
    Pellets are dispensed onto a foraging wheel once the mouse has spun it for
    a pre-defined programmable distance threshold using its forepaws (fictive
    digging).
    %
    The arena contains up to six scale-equipped nesting modules that allows
    housing of mice in the arena and weight monitoring.
    %
    Behavioural monitoring is achieved by an array of high-speed cameras (up to
    15), by which mouse location, mouse identity and body parts can be track in
    real time.
    %
    }
\label{fig:arena}
\end{figure}


\subsubsection{Data sharing}

The large dataset sizes generated by NaLoDuCo experiments, on the order of
hundreads of terabytes,  make it impractical to distribute data to users, and
require to bring users to data. Fortunately, cloud technologies are now mature
to allows this.
%
We will store data in the Distributed Archives for Neuroscience Data
Integration (DANDI), which uses Amazon S3 buckets, and we will provide software
to visualize and analyze data in Amazon EC2 instances, to avoid costly data
transfers.

CatalystNeuro has played a pivotal role in supporting the development and
operations of DANDI.

\subsubsection{Data visualisation}

Visualisations are essential for scientific discovery. Our visualisation tools
need to display very large datasets at different temporal scales, from
milliseconds to weeks and months, and they need to be web based.
%
We will use multi-resolution visualization techniques, which store data at
various resolutions, and use the approriate resolution for each zoom level.
%
Web-based visualisation will be optimized using web workers
\citep{webWorkers}.

Dr.~Magland has extensive experience buidling web based visualizations (e.g.,
Neurosift \citep{neurosift})

\subsubsection{Spike sorting}

Spike sorting is specially challenging in NaLoDuCo experimentation since we
want to track individual neurons of freely moving mice for weeks to months.
%
In addition, spike sorting needs to be done online, to allow experiments driven
by real-time machine learning inference, as described below.
%
Furthermore, sorting algorithms need to operate on probes with hundreads of
channels.

Methods have been proposed for tracking neurons for long periods of time
\citep[e.g.,][]{yuanEtAl24,vanBeestEtAl24} and for online sorting
\citep[e.g.,][]{rutishauserEtAl06,santhanamEtAl04}. We will rigorously evaluate these
methods and report the results of these evaluation, so that researchers can
choose the method that best fits their needs.

Prof.~Sahani pioneered the use of Bayesian inference methods for spike sorting
\citep{sahani99}. Dr. Jeremy Magland has significantly advanced the field of
spike sorting, particularly through his development of MountainSort
\citep{mountainSort5} and his contributions to
SpikeInterface\citep{spikeInterface}.

\subsubsection{Data analysis}

The very large size of NaLoDuCo experimental data, the fact that the statistics
of these data change across time, and the requirement for real-time and
close-loop inference create new challenges to conventional machine learning
methods.

We have selected a few data analysis problems in NaLoDuCo experimentation to
address in this project (Figure~\ref{fig:dataAnalysis}).
%
For each of these problems we will evaluate a few existing machine learning
methods and, if needed, develop new ones.
%
We will document the results of these evaluations, and create benchmarks
comparing methods performance.
%
We will create a resource where scientists performing NaLoDuCo experimentation
can use to choose the data analysis methods that best fits there needs.

For behavioral data, we will evaluate methods to track multiple body parts of
animals \citep[e.g.,][and a switching-linear-dyanamical method using RFIDs that
we will develop]{mathisEtAl18,pereiraEtAl22,bidermanEtAl24},
%
to infer kinematics of foraging mice \citep[e.g.,][]{ldspython,challaEtAl11},
%
to segment behavior into discrete states \citep[e.g.,][and a hierarchical HMM
that we will develop]{wiltschkoEtAl15,hsuAndYttri21},
%
and to infer the rules that govern mice behavior from behavioral observations
only (i.e., policy inference) \citep[e.g.,][]{ziebartEtAl08,zhuEtAl23}.

For neural data, we will evaluate methods to estimate low-dimensional continual
representations of high-dimensional spiking activity (i.e., latents inference)
\citep[e.g.,][]{mackeEtAl11,dunckerAndSahani18,pandarinathEtAl18,saniEtAl21},
%
to segment neural activity into discrete states
\citep[e.g.,][]{chenEtAl09,escolaEtAl11},
%
and to decode environment variables from neural activity
\citep[e.g.,][]{dengEtAl15,kloostermanEtAl14,tampuuEtAl19}.

\subsubsection{Inference-driven experimentation}

We call inference-driven experimentation to a type of experimentation driven by
machine learning inferences on neural or behavioral data, where the result of
these inferences can change the experiment in real time.

We will apply inference-driven experimentation to test if patterns of neural
activity are causally related to foraging behaviors.
%
We would first check that a pattern of neural activity a brain region always
precedes a given foraging behavior. We would then detect the occurence of the
pattern and in real time optogenetically inactivate the brain region.
If the behavior dissapears the causality argument would be supported.

We will use the Bonsai ecosystem for experimental control~\citep{bonsai} and
online machine learning functionality that, funded by a BBSRC
award~\citep{bbsrcAward}, we are adding to Bonsai~\citep{bonsaiML}.

